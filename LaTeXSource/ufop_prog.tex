\documentclass[11pt]{article}
\usepackage[utf8]{inputenc}
\usepackage[T1]{fontenc}
\usepackage[brazil]{babel}
\usepackage{geometry}
\geometry{a4paper,margin=2cm}
\usepackage{longtable,graphicx,multirow,enumitem,tabularx,setspace,ragged2e}
\setlist{noitemsep,leftmargin=*}
\renewcommand\arraystretch{1.15}
% Caixa de seleção
\newcommand{\chk}[1]{\ifx#1X\setlength\fboxsep{1pt}\fbox{\rule{1.3ex}{0pt}X}%
\else\setlength\fboxsep{1pt}\fbox{\rule{1.3ex}{0pt}\rule{1.3ex}{0pt}}\fi}
\pagestyle{empty}
\begin{document}
\begin{center}
\begin{tabular}{lccr}
 \multirow{3}{*}{\includegraphics[height=2.7cm]{brasao.png}} &
 \multicolumn{2}{c}{\bfseries UNIVERSIDADE FEDERAL DE OURO PRETO} &
 \ \ \ \ \multirow{3}{*}{\includegraphics[height=2.7cm]{ufop.png}} \\
 & \multicolumn{2}{c}{\bfseries PRÓ-REITORIA DE GRADUAÇÃO} & \\
 & \multicolumn{2}{c}{} & \\
 & \multicolumn{2}{c}{\Large\bfseries PROGRAMA DE DISCIPLINA} & \\
\end{tabular}
\end{center}

\begin{center}
\begin{longtable}{|p{4cm}|p{4cm}|p{4cm}|p{4cm}|}
\hline
\multicolumn{3}{|p{12cm}|}{Nome do Componente Curricular em Português:} &
\multicolumn{1}{p{4cm}|}{Código:} \\ 
\multicolumn{3}{|p{12cm}|}{\textbf{Matemática Discreta I}} &
\textbf{BCC101}\\ 
\multicolumn{3}{|p{12cm}|}{Nome do Componente Curricular em Inglês:} & \\ 
\multicolumn{3}{|p{12cm}|}{\textbf{Discrete Mathematics I}} & \\ 
\hline
\multicolumn{3}{|p{12cm}|}{Nome e Sigla do Departamento} & Unidade Acadêmica: \\ 
\multicolumn{3}{|p{12cm}|}{Departamento de Computação (DECOM)} & {ICEB} \\ 
\hline
\multicolumn{4}{|p{16cm}|}{Modalidade de Oferta:
[X] presencial \hspace{1cm}
[ ] à distância}\\
\hline
\multicolumn{2}{|p{8cm}|}{Carga horária semestral} &
\multicolumn{2}{p{8cm}|}{Carga horária semanal}\\
\hline
\multicolumn{1}{|p{4cm}|}{Total} &
\multicolumn{1}{p{4cm}|}{Extensionista} &
\multicolumn{1}{p{4cm}|}{Teórica} &
\multicolumn{1}{p{4cm}|}{Prática} \\ 
\multicolumn{1}{|p{4cm}|}{60\,horas} &
\multicolumn{1}{p{4cm}|}{0\;horas} &
\multicolumn{1}{p{4cm}|}{4\;horas/aula} &
\multicolumn{1}{p{4cm}|}{0\;horas/aula} \\ 
\hline
\multicolumn{4}{|p{16cm}|}{Ementa:}\\
\multicolumn{4}{|p{16cm}|}{}\\
\multicolumn{4}{|p{16cm}|}{Introdução à teoria de conjuntos: definições de conjuntos, operações sobre conjuntos, cardinalidade de conjuntos. Funções: conceitos básicos, composição, funções recursivas. Lógica proposicional e lógica de predicados: sintaxe, semântica e sistema de dedução. Estratégias de prova. Indução e recursão.}\\
\multicolumn{4}{|p{16cm}|}{}\\
\hline
\multicolumn{4}{|p{16cm}|}{Conteúdo programático:}\\
\multicolumn{4}{|p{16cm}|}{%
\begin{enumerate}\item Introdução e Revisão de Teoria de Conjuntos
\item Sintaxe e Semântica da Lógica Proposicional
\item Sistema de Dedução da Lógica Proposicional
\item Álgebra Booleana
\item Sintaxe e Semântica da Lógica de Predicados
\item Sistema de Dedução - Lógica de Predicados
\item Álgebra de Predicados
\item Estratégias de prova
\item Indução e Recursão
\item Provas e correção de provas\end{enumerate}}\\
\multicolumn{4}{|p{16cm}|}{}\\
\hline
\multicolumn{4}{|p{16cm}|}{Bibliografia Básica:}\\
\multicolumn{4}{|p{16cm}|}{%
\begin{itemize}\item VELLEMAN, Daniel J. How to Prove It: A Structured Approach. Cambridge: Cambridge University Press, 2006.
\item ROSEN, Kenneth H. Matemática Discreta e suas Aplicações. 6. ed. São Paulo: McGraw-Hill, 2009.
\item O´DONNELL, John; HALL, Cordelia; PAGE, Rex. Discrete Mathematics Using a Computer. Glasgow: Springer-Verlag, 2000.\end{itemize}}\\
\multicolumn{4}{|p{16cm}|}{}\\
\hline
\multicolumn{4}{|p{16cm}|}{Bibliografia Complementar:}\\
\multicolumn{4}{|p{16cm}|}{%
\begin{itemize}\item HUTH, Michael; RYAN, Mark. Lógica em Ciência da Computação: Modelagem e Argumentação sobre Sistemas. 2. ed. Rio de Janeiro: LTC, 2008.
\item SCHEINERMAN, Edward R. Matemática Discreta: Uma Introdução. São Paulo: Cengage Learning, 2011.
\item GERSTING, Judith L. Fundamentos Matemáticos para a Ciência da Computação. 5. ed. Rio de Janeiro: LTC, 2004.\end{itemize}}\\
\hline
\end{longtable}
\end{center}

\clearpage
\begin{center}
\begin{tabular}{lccr}
 \multirow{3}{*}{\includegraphics[height=2.7cm]{brasao.png}} &
 \multicolumn{2}{c}{\bfseries UNIVERSIDADE FEDERAL DE OURO PRETO} &
 \ \ \ \ \multirow{3}{*}{\includegraphics[height=2.7cm]{ufop.png}} \\
 & \multicolumn{2}{c}{\bfseries PRÓ-REITORIA DE GRADUAÇÃO} & \\
 & \multicolumn{2}{c}{} & \\
 & \multicolumn{2}{c}{\Large\bfseries PROGRAMA DE DISCIPLINA} & \\
\end{tabular}
\end{center}

\begin{center}
\begin{longtable}{|p{4cm}|p{4cm}|p{4cm}|p{4cm}|}
\hline
\multicolumn{3}{|p{12cm}|}{Nome do Componente Curricular em Português:} &
\multicolumn{1}{p{4cm}|}{Código:} \\ 
\multicolumn{3}{|p{12cm}|}{\textbf{Introdução à Programação}} &
\textbf{BCC201}\\ 
\multicolumn{3}{|p{12cm}|}{Nome do Componente Curricular em Inglês:} & \\ 
\multicolumn{3}{|p{12cm}|}{\textbf{Introduction to Programming}} & \\ 
\hline
\multicolumn{3}{|p{12cm}|}{Nome e Sigla do Departamento} & Unidade Acadêmica: \\ 
\multicolumn{3}{|p{12cm}|}{Departamento de Computação - DECOM} & {ICEB} \\ 
\hline
\multicolumn{4}{|p{16cm}|}{Modalidade de Oferta:
[X] presencial \hspace{1cm}
[ ] à distância}\\
\hline
\multicolumn{2}{|p{8cm}|}{Carga horária semestral} &
\multicolumn{2}{p{8cm}|}{Carga horária semanal}\\
\hline
\multicolumn{1}{|p{4cm}|}{Total} &
\multicolumn{1}{p{4cm}|}{Extensionista} &
\multicolumn{1}{p{4cm}|}{Teórica} &
\multicolumn{1}{p{4cm}|}{Prática} \\ 
\multicolumn{1}{|p{4cm}|}{90\,horas} &
\multicolumn{1}{p{4cm}|}{0\;horas} &
\multicolumn{1}{p{4cm}|}{4\;horas/aula} &
\multicolumn{1}{p{4cm}|}{2\;horas/aula} \\ 
\hline
\multicolumn{4}{|p{16cm}|}{Ementa:}\\
\multicolumn{4}{|p{16cm}|}{}\\
\multicolumn{4}{|p{16cm}|}{Introdução à lógica de programação; conceitos básicos sobre algoritmos, utilização e formas de representação (fluxograma e portugol); tipos de dados; variáveis e constantes; expressões e operadores relacionais, aritméticos e lógicos; estruturas condicionais e de repetição; sub-programação: modularização de programas (funções e procedimentos); estruturas de dados homogêneas (vetores e matrizes) e heterogêneas (estruturas/registro); manipulação de cadeias de caracteres; ponteiros; alocação dinâmica de memória; processamento de arquivos.}\\
\multicolumn{4}{|p{16cm}|}{}\\
\hline
\multicolumn{4}{|p{16cm}|}{Conteúdo programático:}\\
\multicolumn{4}{|p{16cm}|}{%
\begin{enumerate}\item Representação de dados
\item Conceitos e Representação de algoritmos
\item Fluxograma e portugol
\item Conceitos básicos de programação, valores, tipos e expressões
\item Variáveis, comandos de atribuição e de entrada e saída
\item Comandos de controle de fluxo
\item Comando de decisão (if)
\item Comandos de decisão múltipla, de salto (switch, break)
\item Comando de repetição (while, do-while, for)
\item Sub-programação: Funções; procedimentos e parâmetros
\item Estruturas de dados homogêneas (vetores)
\item Cadeia de caracteres (strings)
\item Estruturas de dados homogêneas (Matrizes)
\item Estrutura heterogêneas
\item Apontadores e memória dinâmica (Ponteiros)
\item Arquivos\end{enumerate}}\\
\multicolumn{4}{|p{16cm}|}{}\\
\hline
\multicolumn{4}{|p{16cm}|}{Bibliografia Básica:}\\
\multicolumn{4}{|p{16cm}|}{%
\begin{itemize}\item DEITEL, P.; DEITEL, H. M. C: como programar. 6. ed. São Paulo: Pearson Education, 2011.
\item DEITEL, H. M.; DEITEL, P. J. C++: como programar. 5. ed. São Paulo: Pearson Prentice Hall, 2006.
\item SOUZA, M. A. F. de. Algoritmos e lógica de programação. São Paulo: Cengage Learning, 2005.\end{itemize}}\\
\multicolumn{4}{|p{16cm}|}{}\\
\hline
\multicolumn{4}{|p{16cm}|}{Bibliografia Complementar:}\\
\multicolumn{4}{|p{16cm}|}{%
\begin{itemize}\item ASCENCIO, A. F. G.; CAMPOS, E. A. V. Fundamentos da programação de computadores: algoritmos, pascal e c/c++. 2. ed. São Paulo: Prentice-Hall, 2007.
\item GUEDES, S. Lógica de Programação Algorítmica. São Paulo: Pearson Education do Brasil, 2014.
\item MIZRAHI, V. V. Treinamento em linguagem C. 2. ed. São Paulo: Pearson Education, 2010.
\item MIZRAHI, V. V. Treinamento em linguagem C++: módulo 2. São Paulo: Pearson Education, 2006.
\item SAVITCH, W. J. C++ absoluto. São Paulo: Pearson Education: Addison Wesley, 2004.\end{itemize}}\\
\hline
\end{longtable}
\end{center}

\clearpage
\end{document}