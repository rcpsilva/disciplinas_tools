\documentclass[11pt]{article}
\usepackage[utf8]{inputenc}
\usepackage[T1]{fontenc}
\usepackage[brazil]{babel}
\usepackage{geometry}
\geometry{a4paper,margin=2cm}
\usepackage{longtable,graphicx,multirow,enumitem,tabularx,setspace,ragged2e}
\setlist{noitemsep,leftmargin=*}
\renewcommand\arraystretch{1.15}
% Caixa de seleção
\newcommand{\chk}[1]{\ifx#1X\setlength\fboxsep{1pt}\fbox{\rule{1.3ex}{0pt}X}%
\else\setlength\fboxsep{1pt}\fbox{\rule{1.3ex}{0pt}\rule{1.3ex}{0pt}}\fi}
\pagestyle{empty}
\begin{document}
\begin{center}
\begin{tabular}{lccr}
 \multirow{3}{*}{\includegraphics[height=2.7cm]{brasao.png}} &
 \multicolumn{2}{c}{\bfseries UNIVERSIDADE FEDERAL DE OURO PRETO} &
 \ \ \ \ \multirow{3}{*}{\includegraphics[height=2.7cm]{ufop.png}} \\
 & \multicolumn{2}{c}{\bfseries PRÓ-REITORIA DE GRADUAÇÃO} & \\
 & \multicolumn{2}{c}{} & \\
 & \multicolumn{2}{c}{\Large\bfseries PROGRAMA DE DISCIPLINA} & \\
\end{tabular}
\end{center}

\begin{center}
\begin{longtable}{|p{4cm}|p{4cm}|p{4cm}|p{4cm}|}
\hline
\multicolumn{3}{|p{12cm}|}{Nome do Componente Curricular em Português:} &
\multicolumn{1}{p{4cm}|}{Código:} \\ 
\multicolumn{3}{|p{12cm}|}{\textbf{Programação Orientada a Objetos}} &
\textbf{BCC138}\\ 
\multicolumn{3}{|p{12cm}|}{Nome do Componente Curricular em Inglês:} & \\ 
\multicolumn{3}{|p{12cm}|}{\textbf{Object Oriented Programming}} & \\ 
\hline
\multicolumn{3}{|p{12cm}|}{Nome e Sigla do Departamento} & Unidade Acadêmica: \\ 
\multicolumn{3}{|p{12cm}|}{Departamento de Computação (DECOM)} & {ICEB} \\ 
\hline
\multicolumn{4}{|p{16cm}|}{Modalidade de Oferta:
[X] presencial \hspace{1cm}
[ ] à distância}\\
\hline
\multicolumn{2}{|p{8cm}|}{Carga horária semestral} &
\multicolumn{2}{p{8cm}|}{Carga horária semanal}\\
\hline
\multicolumn{1}{|p{4cm}|}{Total} &
\multicolumn{1}{p{4cm}|}{Extensionista} &
\multicolumn{1}{p{4cm}|}{Teórica} &
\multicolumn{1}{p{4cm}|}{Prática} \\ 
\multicolumn{1}{|p{4cm}|}{60\,horas} &
\multicolumn{1}{p{4cm}|}{0\;horas} &
\multicolumn{1}{p{4cm}|}{4\;horas/aula} &
\multicolumn{1}{p{4cm}|}{0\;horas/aula} \\ 
\hline
\multicolumn{4}{|p{16cm}|}{Ementa:}\\
\multicolumn{4}{|p{16cm}|}{}\\
\multicolumn{4}{|p{16cm}|}{Conceitos básicos de orientação a objetos, Classe, Objeto, Mensagem, Encapsulamento, Herança, Polimorfismo, Ligação dinâmica, Tratamento de exceções, Genéricos, Coleções, Modelagem UML, Interface gráfica em ambientes orientados a objetos. Objetos persistentes.}\\
\multicolumn{4}{|p{16cm}|}{}\\
\hline
\multicolumn{4}{|p{16cm}|}{Conteúdo programático:}\\
\multicolumn{4}{|p{16cm}|}{%
\begin{enumerate}\item Visão geral do paradigma de programação orientada a objetos
\item Modelagem UML
\item Classes, objetos, mensagens
\item Herança
\item Polimorfismo
\item Ligação dinâmica
\item Tratamento de exceções
\item Genéricos
\item Coleções
\item Objetos persistentes
\item Interfaces gráficas
\item Programação em C++ e Java\end{enumerate}}\\
\multicolumn{4}{|p{16cm}|}{}\\
\hline
\multicolumn{4}{|p{16cm}|}{Bibliografia Básica:}\\
\multicolumn{4}{|p{16cm}|}{%
\begin{itemize}\item DEITEL, Harvey M.; DEITEL, P. J. C++ como programar. 5. ed. São Paulo: Pearson Prentice Hall, 2006.
\item DEITEL, Paul; DEITEL, Paul J. Java: como programar. 8. ed. São Paulo: Prentice-Hall, 2010.
\item FOWLER, Martin. UML Essencial. Um breve guia para a linguagem-padrão de modelagem de objetos. 3. ed. Bookman, 2005.
\item MEYER, Bertrand. Object-oriented software construction. 2nd. ed. Upper Saddle River, NJ: Prentice-Hall PTR, 1997.
\item BOOCH, Grady. Object-oriented analysis and design with applications. 3rd. ed. New Delhi: Pearson, 2009.\end{itemize}}\\
\multicolumn{4}{|p{16cm}|}{}\\
\hline
\multicolumn{4}{|p{16cm}|}{Bibliografia Complementar:}\\
\multicolumn{4}{|p{16cm}|}{%
\begin{itemize}\item LEE, Richard C.; TEPFENHART, William M. UML e C++: guia prático de desenvolvimento orientado a objeto. Pearson, 2001.
\item PAGE-JONES, Meilir. Fundamentos do desenho orientado a objeto com UML. São Paulo: Makron Books, 2001.
\item ORGANIZADOR, Rafael Felix. Programação orientada a objetos. Pearson, 2017.
\item SINTES, Anthony. Aprenda Programação orientada a objetos em 21 dias. 5. ed. Pearson, 2014.
\item SANTOS, Rafael. Introdução à programação orientada a objetos usando JAVA. 9. ed. Rio de Janeiro: Elsevier, 2003.
\item BARNES, David J.; KÖLLING, Michael. Programação orientada a objetos com Java: uma introdução prática usando o BlueJ. 4. ed. São Paulo: Prentice-Hall, 2009.
\item LARMAN, Craig. Utilizando UML e padrões: uma introdução à análise e ao projeto orientados a objetos. 2. ed. Porto Alegre: Bookman, 2004.\end{itemize}}\\
\hline
\end{longtable}
\end{center}

\clearpage
\begin{center}
\begin{tabular}{lccr}
 \multirow{3}{*}{\includegraphics[height=2.7cm]{brasao.png}} &
 \multicolumn{2}{c}{\bfseries UNIVERSIDADE FEDERAL DE OURO PRETO} &
 \ \ \ \ \multirow{3}{*}{\includegraphics[height=2.7cm]{ufop.png}} \\
 & \multicolumn{2}{c}{\bfseries PRÓ-REITORIA DE GRADUAÇÃO} & \\
 & \multicolumn{2}{c}{} & \\
 & \multicolumn{2}{c}{\Large\bfseries PROGRAMA DE DISCIPLINA} & \\
\end{tabular}
\end{center}

\begin{center}
\begin{longtable}{|p{4cm}|p{4cm}|p{4cm}|p{4cm}|}
\hline
\multicolumn{3}{|p{12cm}|}{Nome do Componente Curricular em Português:} &
\multicolumn{1}{p{4cm}|}{Código:} \\ 
\multicolumn{3}{|p{12cm}|}{\textbf{Estrutura de Dados I}} &
\textbf{BCC137}\\ 
\multicolumn{3}{|p{12cm}|}{Nome do Componente Curricular em Inglês:} & \\ 
\multicolumn{3}{|p{12cm}|}{\textbf{Data Structures I}} & \\ 
\hline
\multicolumn{3}{|p{12cm}|}{Nome e Sigla do Departamento} & Unidade Acadêmica: \\ 
\multicolumn{3}{|p{12cm}|}{Departamento de Computação (DECOM)} & {ICEB} \\ 
\hline
\multicolumn{4}{|p{16cm}|}{Modalidade de Oferta:
[X] presencial \hspace{1cm}
[ ] à distância}\\
\hline
\multicolumn{2}{|p{8cm}|}{Carga horária semestral} &
\multicolumn{2}{p{8cm}|}{Carga horária semanal}\\
\hline
\multicolumn{1}{|p{4cm}|}{Total} &
\multicolumn{1}{p{4cm}|}{Extensionista} &
\multicolumn{1}{p{4cm}|}{Teórica} &
\multicolumn{1}{p{4cm}|}{Prática} \\ 
\multicolumn{1}{|p{4cm}|}{90\,horas} &
\multicolumn{1}{p{4cm}|}{0\;horas} &
\multicolumn{1}{p{4cm}|}{4\;horas/aula} &
\multicolumn{1}{p{4cm}|}{2\;horas/aula} \\ 
\hline
\multicolumn{4}{|p{16cm}|}{Ementa:}\\
\multicolumn{4}{|p{16cm}|}{}\\
\multicolumn{4}{|p{16cm}|}{Recursividade; conceitos básicos de análise assintótica de algoritmos; tipos abstratos de dados; estruturas de dados: listas, pilhas, filas de prioridade e árvores binárias; algoritmos de ordenação por comparação de chaves (seleção, inserção, bolha, shellsort, quicksort, mergesort, heapsort); algoritmos de ordenação em tempo linear (counting sort, radix sort e bucket sort); algoritmos de pesquisa (simples, binária, árvores binárias de busca, hashing, conjuntos e mapas).}\\
\multicolumn{4}{|p{16cm}|}{}\\
\hline
\multicolumn{4}{|p{16cm}|}{Conteúdo programático:}\\
\multicolumn{4}{|p{16cm}|}{%
\begin{enumerate}\item Revisão de alocação dinâmica de memória
\item Recursividade
\item Noções de análise de complexidade de algoritmos
\item  -  Conceitos
\item  -  Medidas de avaliação: tempo e espaço
\item  - Análise assintótica: notação $O$, $\Omega$ e $\Theta$
\item  -  Hierarquia de funções e classes de problemas
\item Tipos de dados abstratos
\item Estruturas de dados
\item  - Listas
\item  - Pilhas
\item  - Filas
\item  - Filas de prioridade
\item  - Árvores
\item  - Conjuntos
\item  - Mapas
\item Algoritmos
\item  - Ordenação por comparação: Selection Sort, Insertion Sort, Bubblesort, Shellsort, Quicksort, Heapsort e Mergesort
\item  - Ordenação em tempo linear: Counting Sort, Radix Sort e Bucket Sort
\item  - Pesquisa: Simples, Binária, Árvores Binárias, AVL e Hashing\end{enumerate}}\\
\multicolumn{4}{|p{16cm}|}{}\\
\hline
\multicolumn{4}{|p{16cm}|}{Bibliografia Básica:}\\
\multicolumn{4}{|p{16cm}|}{%
\begin{itemize}\item ZIVIANI, N. Projeto de algoritmos: com implementações em Pascal e C. 3. ed. rev. e ampl. São Paulo: Cengage Learning, 2011.
\item CELES, W.; CERQUEIRA, R.; RANGEL, J. L. Introdução a Estruturas de Dados: com técnicas de programação em C. Rio de Janeiro: Elsevier, 2004.
\item CORMEN, T. H.; LEISERSON, C. E.; RIVEST, R. L.; STEIN, C. Algoritmos: teoria e prática. Rio de Janeiro: Campus, 2002.\end{itemize}}\\
\multicolumn{4}{|p{16cm}|}{}\\
\hline
\multicolumn{4}{|p{16cm}|}{Bibliografia Complementar:}\\
\multicolumn{4}{|p{16cm}|}{%
\begin{itemize}\item KLEINBERG, J.; TARDOS, E. Algorithm Design. Boston: Addison-Wesley, 2006.
\item KNUTH, D. E. The Art of Computer Programming. Upper Saddle River: Addison-Wesley, 2005.
\item GOODRICH, M. T.; TAMASSIA, R.; COPSTEIN, B. Projeto de algoritmos: fundamentos, análise e exemplos da Internet. Porto Alegre: Bookman, 2004.
\item DROZDEK, A. Estrutura de Dados e Algoritmos em C++. São Paulo: Cengage Learning, 2002.
\item TENENBAUM, A. M.; LANGSAM, Y.; AUGENSTEIN, M. Estruturas de Dados usando C. São Paulo: Makron Books, 1995.\end{itemize}}\\
\hline
\end{longtable}
\end{center}

\clearpage
\begin{center}
\begin{tabular}{lccr}
 \multirow{3}{*}{\includegraphics[height=2.7cm]{brasao.png}} &
 \multicolumn{2}{c}{\bfseries UNIVERSIDADE FEDERAL DE OURO PRETO} &
 \ \ \ \ \multirow{3}{*}{\includegraphics[height=2.7cm]{ufop.png}} \\
 & \multicolumn{2}{c}{\bfseries PRÓ-REITORIA DE GRADUAÇÃO} & \\
 & \multicolumn{2}{c}{} & \\
 & \multicolumn{2}{c}{\Large\bfseries PROGRAMA DE DISCIPLINA} & \\
\end{tabular}
\end{center}

\begin{center}
\begin{longtable}{|p{4cm}|p{4cm}|p{4cm}|p{4cm}|}
\hline
\multicolumn{3}{|p{12cm}|}{Nome do Componente Curricular em Português:} &
\multicolumn{1}{p{4cm}|}{Código:} \\ 
\multicolumn{3}{|p{12cm}|}{\textbf{Cálculo Diferencial e Integral I}} &
\textbf{MTM122}\\ 
\multicolumn{3}{|p{12cm}|}{Nome do Componente Curricular em Inglês:} & \\ 
\multicolumn{3}{|p{12cm}|}{\textbf{Differential and Integral Calculus I}} & \\ 
\hline
\multicolumn{3}{|p{12cm}|}{Nome e Sigla do Departamento} & Unidade Acadêmica: \\ 
\multicolumn{3}{|p{12cm}|}{Departamento de Matemática (DEMAT)} & {Instituto de Ciências Exatas e Biológicas (ICEB)} \\ 
\hline
\multicolumn{4}{|p{16cm}|}{Modalidade de Oferta:
[X] presencial \hspace{1cm}
[ ] à distância}\\
\hline
\multicolumn{2}{|p{8cm}|}{Carga horária semestral} &
\multicolumn{2}{p{8cm}|}{Carga horária semanal}\\
\hline
\multicolumn{1}{|p{4cm}|}{Total} &
\multicolumn{1}{p{4cm}|}{Extensionista} &
\multicolumn{1}{p{4cm}|}{Teórica} &
\multicolumn{1}{p{4cm}|}{Prática} \\ 
\multicolumn{1}{|p{4cm}|}{90\,horas} &
\multicolumn{1}{p{4cm}|}{0\;horas} &
\multicolumn{1}{p{4cm}|}{6\;horas/aula} &
\multicolumn{1}{p{4cm}|}{0\;horas/aula} \\ 
\hline
\multicolumn{4}{|p{16cm}|}{Ementa:}\\
\multicolumn{4}{|p{16cm}|}{}\\
\multicolumn{4}{|p{16cm}|}{Números reais; funções; limites; continuidade; derivada e aplicações; a integral.}\\
\multicolumn{4}{|p{16cm}|}{}\\
\hline
\multicolumn{4}{|p{16cm}|}{Conteúdo programático:}\\
\multicolumn{4}{|p{16cm}|}{%
\begin{enumerate}\item Números Reais: conjuntos numéricos; propriedades e operações; inequações; valor absoluto.
\item Funções e Gráficos: função de primeiro grau; de segundo grau; funções trigonométricas, exponencial, hiperbólicas, compostas e inversas.
\item Limite, Continuidade e Derivada: definição de limite, continuidade; limites laterais, no infinito e infinitos; propriedades e limites fundamentais; funções deriváveis; retas tangentes e normais; diferencial.
\item Funções e suas Derivadas: regras de derivação; derivada das funções trigonométricas, exponencial, inversa, trigonométricas inversas e logarítmica.
\item Aplicações da Derivada: máximos e mínimos; Teorema do Valor Médio; regra de L’Hospital; crescimento e concavidade; gráficos de funções; problemas de otimização; taxa de variação.
\item A Integral: integral indefinida e suas propriedades; integral definida; área de regiões planas; Teorema Fundamental do Cálculo.
\item Técnicas de Integração: substituição; partes; frações parciais; potências e produtos de funções trigonométricas; substituições inversas.\end{enumerate}}\\
\multicolumn{4}{|p{16cm}|}{}\\
\hline
\multicolumn{4}{|p{16cm}|}{Bibliografia Básica:}\\
\multicolumn{4}{|p{16cm}|}{%
\begin{itemize}\item FLEMMING, Diva Marília; GONÇALVES, Mirian Buss. Cálculo A: funções, limite, derivação, integração. 5. ed. rev. e amp. São Paulo/Florianópolis: Makron Books/Editora da UFSC, 1992.
\item LEITHOLD, Louis. O Cálculo com Geometria Analítica. 3. ed. São Paulo: Harbra, 1994.
\item STEWART, James. Cálculo – Volume I. 6. ed. São Paulo: Cengage Learning, 2010.\end{itemize}}\\
\multicolumn{4}{|p{16cm}|}{}\\
\hline
\multicolumn{4}{|p{16cm}|}{Bibliografia Complementar:}\\
\multicolumn{4}{|p{16cm}|}{%
\begin{itemize}\item ANTON, Howard. Cálculo: um novo horizonte – Vol. 1. 6. ed. Porto Alegre: Bookman, 2000.
\item GUIDORIZZI, Hamilton Luiz. Um Curso de Cálculo – Vol. 1. 5. ed. São Paulo: LTC, 2001.
\item MUNEM, Mustafa A.; FOULIS, David J. Cálculo – Volume 1. Rio de Janeiro: Guanabara Koogan, 1982.
\item SIMMONS, George Finlay. Cálculo com Geometria Analítica – Volume 1. São Paulo: Makron Books, 1987.
\item THOMAS, George B.; HASS, Joel; WEIR, Maurice D. Cálculo – Volume 1. 12. ed. São Paulo: Pearson Education do Brasil, 2013.\end{itemize}}\\
\hline
\end{longtable}
\end{center}

\clearpage
\begin{center}
\begin{tabular}{lccr}
 \multirow{3}{*}{\includegraphics[height=2.7cm]{brasao.png}} &
 \multicolumn{2}{c}{\bfseries UNIVERSIDADE FEDERAL DE OURO PRETO} &
 \ \ \ \ \multirow{3}{*}{\includegraphics[height=2.7cm]{ufop.png}} \\
 & \multicolumn{2}{c}{\bfseries PRÓ-REITORIA DE GRADUAÇÃO} & \\
 & \multicolumn{2}{c}{} & \\
 & \multicolumn{2}{c}{\Large\bfseries PROGRAMA DE DISCIPLINA} & \\
\end{tabular}
\end{center}

\begin{center}
\begin{longtable}{|p{4cm}|p{4cm}|p{4cm}|p{4cm}|}
\hline
\multicolumn{3}{|p{12cm}|}{Nome do Componente Curricular em Português:} &
\multicolumn{1}{p{4cm}|}{Código:} \\ 
\multicolumn{3}{|p{12cm}|}{\textbf{Matemática Discreta I}} &
\textbf{BCC101}\\ 
\multicolumn{3}{|p{12cm}|}{Nome do Componente Curricular em Inglês:} & \\ 
\multicolumn{3}{|p{12cm}|}{\textbf{Discrete Mathematics I}} & \\ 
\hline
\multicolumn{3}{|p{12cm}|}{Nome e Sigla do Departamento} & Unidade Acadêmica: \\ 
\multicolumn{3}{|p{12cm}|}{Departamento de Computação (DECOM)} & {ICEB} \\ 
\hline
\multicolumn{4}{|p{16cm}|}{Modalidade de Oferta:
[X] presencial \hspace{1cm}
[ ] à distância}\\
\hline
\multicolumn{2}{|p{8cm}|}{Carga horária semestral} &
\multicolumn{2}{p{8cm}|}{Carga horária semanal}\\
\hline
\multicolumn{1}{|p{4cm}|}{Total} &
\multicolumn{1}{p{4cm}|}{Extensionista} &
\multicolumn{1}{p{4cm}|}{Teórica} &
\multicolumn{1}{p{4cm}|}{Prática} \\ 
\multicolumn{1}{|p{4cm}|}{60\,horas} &
\multicolumn{1}{p{4cm}|}{0\;horas} &
\multicolumn{1}{p{4cm}|}{4\;horas/aula} &
\multicolumn{1}{p{4cm}|}{0\;horas/aula} \\ 
\hline
\multicolumn{4}{|p{16cm}|}{Ementa:}\\
\multicolumn{4}{|p{16cm}|}{}\\
\multicolumn{4}{|p{16cm}|}{Introdução à teoria de conjuntos: definições de conjuntos, operações sobre conjuntos, cardinalidade de conjuntos. Funções: conceitos básicos, composição, funções recursivas. Lógica proposicional e lógica de predicados: sintaxe, semântica e sistema de dedução. Estratégias de prova. Indução e recursão.}\\
\multicolumn{4}{|p{16cm}|}{}\\
\hline
\multicolumn{4}{|p{16cm}|}{Conteúdo programático:}\\
\multicolumn{4}{|p{16cm}|}{%
\begin{enumerate}\item Introdução e Revisão de Teoria de Conjuntos
\item Sintaxe e Semântica da Lógica Proposicional
\item Sistema de Dedução da Lógica Proposicional
\item Álgebra Booleana
\item Sintaxe e Semântica da Lógica de Predicados
\item Sistema de Dedução - Lógica de Predicados
\item Álgebra de Predicados
\item Estratégias de prova
\item Indução e Recursão
\item Provas e correção de provas\end{enumerate}}\\
\multicolumn{4}{|p{16cm}|}{}\\
\hline
\multicolumn{4}{|p{16cm}|}{Bibliografia Básica:}\\
\multicolumn{4}{|p{16cm}|}{%
\begin{itemize}\item VELLEMAN, Daniel J. How to Prove It: A Structured Approach. Cambridge: Cambridge University Press, 2006.
\item ROSEN, Kenneth H. Matemática Discreta e suas Aplicações. 6. ed. São Paulo: McGraw-Hill, 2009.
\item O´DONNELL, John; HALL, Cordelia; PAGE, Rex. Discrete Mathematics Using a Computer. Glasgow: Springer-Verlag, 2000.\end{itemize}}\\
\multicolumn{4}{|p{16cm}|}{}\\
\hline
\multicolumn{4}{|p{16cm}|}{Bibliografia Complementar:}\\
\multicolumn{4}{|p{16cm}|}{%
\begin{itemize}\item HUTH, Michael; RYAN, Mark. Lógica em Ciência da Computação: Modelagem e Argumentação sobre Sistemas. 2. ed. Rio de Janeiro: LTC, 2008.
\item SCHEINERMAN, Edward R. Matemática Discreta: Uma Introdução. São Paulo: Cengage Learning, 2011.
\item GERSTING, Judith L. Fundamentos Matemáticos para a Ciência da Computação. 5. ed. Rio de Janeiro: LTC, 2004.\end{itemize}}\\
\hline
\end{longtable}
\end{center}

\clearpage
\begin{center}
\begin{tabular}{lccr}
 \multirow{3}{*}{\includegraphics[height=2.7cm]{brasao.png}} &
 \multicolumn{2}{c}{\bfseries UNIVERSIDADE FEDERAL DE OURO PRETO} &
 \ \ \ \ \multirow{3}{*}{\includegraphics[height=2.7cm]{ufop.png}} \\
 & \multicolumn{2}{c}{\bfseries PRÓ-REITORIA DE GRADUAÇÃO} & \\
 & \multicolumn{2}{c}{} & \\
 & \multicolumn{2}{c}{\Large\bfseries PROGRAMA DE DISCIPLINA} & \\
\end{tabular}
\end{center}

\begin{center}
\begin{longtable}{|p{4cm}|p{4cm}|p{4cm}|p{4cm}|}
\hline
\multicolumn{3}{|p{12cm}|}{Nome do Componente Curricular em Português:} &
\multicolumn{1}{p{4cm}|}{Código:} \\ 
\multicolumn{3}{|p{12cm}|}{\textbf{Introdução à Programação}} &
\textbf{BCC201}\\ 
\multicolumn{3}{|p{12cm}|}{Nome do Componente Curricular em Inglês:} & \\ 
\multicolumn{3}{|p{12cm}|}{\textbf{Introduction to Programming}} & \\ 
\hline
\multicolumn{3}{|p{12cm}|}{Nome e Sigla do Departamento} & Unidade Acadêmica: \\ 
\multicolumn{3}{|p{12cm}|}{Departamento de Computação - DECOM} & {ICEB} \\ 
\hline
\multicolumn{4}{|p{16cm}|}{Modalidade de Oferta:
[X] presencial \hspace{1cm}
[ ] à distância}\\
\hline
\multicolumn{2}{|p{8cm}|}{Carga horária semestral} &
\multicolumn{2}{p{8cm}|}{Carga horária semanal}\\
\hline
\multicolumn{1}{|p{4cm}|}{Total} &
\multicolumn{1}{p{4cm}|}{Extensionista} &
\multicolumn{1}{p{4cm}|}{Teórica} &
\multicolumn{1}{p{4cm}|}{Prática} \\ 
\multicolumn{1}{|p{4cm}|}{90\,horas} &
\multicolumn{1}{p{4cm}|}{0\;horas} &
\multicolumn{1}{p{4cm}|}{4\;horas/aula} &
\multicolumn{1}{p{4cm}|}{2\;horas/aula} \\ 
\hline
\multicolumn{4}{|p{16cm}|}{Ementa:}\\
\multicolumn{4}{|p{16cm}|}{}\\
\multicolumn{4}{|p{16cm}|}{Introdução à lógica de programação; conceitos básicos sobre algoritmos, utilização e formas de representação (fluxograma e portugol); tipos de dados; variáveis e constantes; expressões e operadores relacionais, aritméticos e lógicos; estruturas condicionais e de repetição; sub-programação: modularização de programas (funções e procedimentos); estruturas de dados homogêneas (vetores e matrizes) e heterogêneas (estruturas/registro); manipulação de cadeias de caracteres; ponteiros; alocação dinâmica de memória; processamento de arquivos.}\\
\multicolumn{4}{|p{16cm}|}{}\\
\hline
\multicolumn{4}{|p{16cm}|}{Conteúdo programático:}\\
\multicolumn{4}{|p{16cm}|}{%
\begin{enumerate}\item Representação de dados
\item Conceitos e Representação de algoritmos
\item Fluxograma e portugol
\item Conceitos básicos de programação, valores, tipos e expressões
\item Variáveis, comandos de atribuição e de entrada e saída
\item Comandos de controle de fluxo
\item Comando de decisão (if)
\item Comandos de decisão múltipla, de salto (switch, break)
\item Comando de repetição (while, do-while, for)
\item Sub-programação: Funções; procedimentos e parâmetros
\item Estruturas de dados homogêneas (vetores)
\item Cadeia de caracteres (strings)
\item Estruturas de dados homogêneas (Matrizes)
\item Estrutura heterogêneas
\item Apontadores e memória dinâmica (Ponteiros)
\item Arquivos\end{enumerate}}\\
\multicolumn{4}{|p{16cm}|}{}\\
\hline
\multicolumn{4}{|p{16cm}|}{Bibliografia Básica:}\\
\multicolumn{4}{|p{16cm}|}{%
\begin{itemize}\item DEITEL, P.; DEITEL, H. M. C: como programar. 6. ed. São Paulo: Pearson Education, 2011.
\item DEITEL, H. M.; DEITEL, P. J. C++: como programar. 5. ed. São Paulo: Pearson Prentice Hall, 2006.
\item SOUZA, M. A. F. de. Algoritmos e lógica de programação. São Paulo: Cengage Learning, 2005.\end{itemize}}\\
\multicolumn{4}{|p{16cm}|}{}\\
\hline
\multicolumn{4}{|p{16cm}|}{Bibliografia Complementar:}\\
\multicolumn{4}{|p{16cm}|}{%
\begin{itemize}\item ASCENCIO, A. F. G.; CAMPOS, E. A. V. Fundamentos da programação de computadores: algoritmos, pascal e c/c++. 2. ed. São Paulo: Prentice-Hall, 2007.
\item GUEDES, S. Lógica de Programação Algorítmica. São Paulo: Pearson Education do Brasil, 2014.
\item MIZRAHI, V. V. Treinamento em linguagem C. 2. ed. São Paulo: Pearson Education, 2010.
\item MIZRAHI, V. V. Treinamento em linguagem C++: módulo 2. São Paulo: Pearson Education, 2006.
\item SAVITCH, W. J. C++ absoluto. São Paulo: Pearson Education: Addison Wesley, 2004.\end{itemize}}\\
\hline
\end{longtable}
\end{center}

\clearpage
\begin{center}
\begin{tabular}{lccr}
 \multirow{3}{*}{\includegraphics[height=2.7cm]{brasao.png}} &
 \multicolumn{2}{c}{\bfseries UNIVERSIDADE FEDERAL DE OURO PRETO} &
 \ \ \ \ \multirow{3}{*}{\includegraphics[height=2.7cm]{ufop.png}} \\
 & \multicolumn{2}{c}{\bfseries PRÓ-REITORIA DE GRADUAÇÃO} & \\
 & \multicolumn{2}{c}{} & \\
 & \multicolumn{2}{c}{\Large\bfseries PROGRAMA DE DISCIPLINA} & \\
\end{tabular}
\end{center}

\begin{center}
\begin{longtable}{|p{4cm}|p{4cm}|p{4cm}|p{4cm}|}
\hline
\multicolumn{3}{|p{12cm}|}{Nome do Componente Curricular em Português:} &
\multicolumn{1}{p{4cm}|}{Código:} \\ 
\multicolumn{3}{|p{12cm}|}{\textbf{Introdução à Inteligência Artificial}} &
\textbf{BIA003}\\ 
\multicolumn{3}{|p{12cm}|}{Nome do Componente Curricular em Inglês:} & \\ 
\multicolumn{3}{|p{12cm}|}{\textbf{Introduction to Artificial Intelligence}} & \\ 
\hline
\multicolumn{3}{|p{12cm}|}{Nome e Sigla do Departamento} & Unidade Acadêmica: \\ 
\multicolumn{3}{|p{12cm}|}{Departamento de Computação (DECOM)} & {ICEB} \\ 
\hline
\multicolumn{4}{|p{16cm}|}{Modalidade de Oferta:
[X] presencial \hspace{1cm}
[ ] à distância}\\
\hline
\multicolumn{2}{|p{8cm}|}{Carga horária semestral} &
\multicolumn{2}{p{8cm}|}{Carga horária semanal}\\
\hline
\multicolumn{1}{|p{4cm}|}{Total} &
\multicolumn{1}{p{4cm}|}{Extensionista} &
\multicolumn{1}{p{4cm}|}{Teórica} &
\multicolumn{1}{p{4cm}|}{Prática} \\ 
\multicolumn{1}{|p{4cm}|}{60\,horas} &
\multicolumn{1}{p{4cm}|}{0\;horas} &
\multicolumn{1}{p{4cm}|}{4\;horas/aula} &
\multicolumn{1}{p{4cm}|}{0\;horas/aula} \\ 
\hline
\multicolumn{4}{|p{16cm}|}{Ementa:}\\
\multicolumn{4}{|p{16cm}|}{}\\
\multicolumn{4}{|p{16cm}|}{O que é Inteligência Artificial; o comportamento do aluno de Inteligência Artificial; áreas de pesquisa do Departamento de Computação; áreas nas quais atuam os profissionais em Inteligência Artificial.}\\
\multicolumn{4}{|p{16cm}|}{}\\
\hline
\multicolumn{4}{|p{16cm}|}{Conteúdo programático:}\\
\multicolumn{4}{|p{16cm}|}{%
\begin{enumerate}\item A área de Inteligência Artificial e suas oportunidades atuais.
\item Áreas de atuação e mercado de trabalho em Inteligência Artificial.
\item O curso de Inteligência Artificial: grade curricular, áreas, relações entre as disciplinas.
\item Organização política da universidade e institutos/unidades.
\item Representação acadêmica: centro acadêmico e movimentos estudantis.
\item Seminários sobre as áreas de pesquisa e extensão dos professores do DECOM.
\item Apresentação das atividades dos laboratórios de pesquisa e extensão.
\item O comportamento do aluno de Inteligência Artificial: organização e conselhos para estudar melhor e ter sucesso no curso.
\item Seminários de profissionais sobre o mercado de trabalho e carreiras na área de Inteligência Artificial.\end{enumerate}}\\
\multicolumn{4}{|p{16cm}|}{}\\
\hline
\multicolumn{4}{|p{16cm}|}{Bibliografia Básica:}\\
\multicolumn{4}{|p{16cm}|}{%
\begin{itemize}\item MORAIS, Regis de (org.). Filosofia da ciência e da tecnologia: introdução metodológica e crítica. 1. ed. Campinas: Papirus, 2013. E-book. Disponível em: https://plataforma.bvirtual.com.br. Acesso em: jun/2025.
\item MEDEIROS, Luciano Frontino de. Inteligência artificial aplicada: uma abordagem introdutória. Curitiba, PR: Intersaberes, 2018. E-book. Disponível em: https://plataforma.bvirtual.com.br. Acesso em: jun/2025.
\item DIAS, Ana Francisca Pinto et al.; GUIMARÃES, João Alexandre Silva Alves; ALVES, Rodrigo Vitorino Souza (org.). Os direitos humanos e a ética na era da inteligência artificial. Indaiatuba, SP: Foco, 2023. E-book. Disponível em: https://plataforma.bvirtual.com.br. Acesso em: jun/2025.\end{itemize}}\\
\multicolumn{4}{|p{16cm}|}{}\\
\hline
\multicolumn{4}{|p{16cm}|}{Bibliografia Complementar:}\\
\multicolumn{4}{|p{16cm}|}{%
\begin{itemize}\item KRELLING NETO, Antonio Osmar. Responsabilidade civil: cibercrimes. 1. ed. São Paulo: Contentus, 2020. E-book. Disponível em: https://plataforma.bvirtual.com.br. Acesso em: jun/2025.
\item TAURION, Cezar. Big data. 1. ed. Rio de Janeiro: Brasport, 2013. E-book. Disponível em: https://plataforma.bvirtual.com.br. Acesso em: jun/2025.
\item MUNHOZ, Antonio Siemsen. Responsabilidade e autoridade social das empresas. 1. ed. Curitiba: Intersaberes, 2015. E-book. Disponível em: https://plataforma.bvirtual.com.br. Acesso em: jun/2025.
\item FLORES, Márcio José das; BESS, Alexandre Leal. Inteligência artificial aplicada a negócios. Curitiba, PR: Intersaberes, 2023. E-book. Disponível em: https://plataforma.bvirtual.com.br. Acesso em: jun/2025.
\item MUNIZ, Antonio et al. Inteligência artificial: entenda como a IA pode impactar no mercado de trabalho e na sociedade. [S.l.]: Brasport, 2024. E-book. Disponível em: https://plataforma.bvirtual.com.br. Acesso em: jun/2025.\end{itemize}}\\
\hline
\end{longtable}
\end{center}

\clearpage
\begin{center}
\begin{tabular}{lccr}
 \multirow{3}{*}{\includegraphics[height=2.7cm]{brasao.png}} &
 \multicolumn{2}{c}{\bfseries UNIVERSIDADE FEDERAL DE OURO PRETO} &
 \ \ \ \ \multirow{3}{*}{\includegraphics[height=2.7cm]{ufop.png}} \\
 & \multicolumn{2}{c}{\bfseries PRÓ-REITORIA DE GRADUAÇÃO} & \\
 & \multicolumn{2}{c}{} & \\
 & \multicolumn{2}{c}{\Large\bfseries PROGRAMA DE DISCIPLINA} & \\
\end{tabular}
\end{center}

\begin{center}
\begin{longtable}{|p{4cm}|p{4cm}|p{4cm}|p{4cm}|}
\hline
\multicolumn{3}{|p{12cm}|}{Nome do Componente Curricular em Português:} &
\multicolumn{1}{p{4cm}|}{Código:} \\ 
\multicolumn{3}{|p{12cm}|}{\textbf{Introdução à Programação}} &
\textbf{BCC201}\\ 
\multicolumn{3}{|p{12cm}|}{Nome do Componente Curricular em Inglês:} & \\ 
\multicolumn{3}{|p{12cm}|}{\textbf{Introduction to Programming}} & \\ 
\hline
\multicolumn{3}{|p{12cm}|}{Nome e Sigla do Departamento} & Unidade Acadêmica: \\ 
\multicolumn{3}{|p{12cm}|}{Departamento de Computação - DECOM} & {ICEB} \\ 
\hline
\multicolumn{4}{|p{16cm}|}{Modalidade de Oferta:
[X] presencial \hspace{1cm}
[ ] à distância}\\
\hline
\multicolumn{2}{|p{8cm}|}{Carga horária semestral} &
\multicolumn{2}{p{8cm}|}{Carga horária semanal}\\
\hline
\multicolumn{1}{|p{4cm}|}{Total} &
\multicolumn{1}{p{4cm}|}{Extensionista} &
\multicolumn{1}{p{4cm}|}{Teórica} &
\multicolumn{1}{p{4cm}|}{Prática} \\ 
\multicolumn{1}{|p{4cm}|}{90\,horas} &
\multicolumn{1}{p{4cm}|}{0\;horas} &
\multicolumn{1}{p{4cm}|}{4\;horas/aula} &
\multicolumn{1}{p{4cm}|}{2\;horas/aula} \\ 
\hline
\multicolumn{4}{|p{16cm}|}{Ementa:}\\
\multicolumn{4}{|p{16cm}|}{}\\
\multicolumn{4}{|p{16cm}|}{Introdução à lógica de programação; conceitos básicos sobre algoritmos, utilização e formas de representação (fluxograma e portugol); tipos de dados; variáveis e constantes; expressões e operadores relacionais, aritméticos e lógicos; estruturas condicionais e de repetição; sub-programação: modularização de programas (funções e procedimentos); estruturas de dados homogêneas (vetores e matrizes) e heterogêneas (estruturas/registro); manipulação de cadeias de caracteres; ponteiros; alocação dinâmica de memória; processamento de arquivos.}\\
\multicolumn{4}{|p{16cm}|}{}\\
\hline
\multicolumn{4}{|p{16cm}|}{Conteúdo programático:}\\
\multicolumn{4}{|p{16cm}|}{%
\begin{enumerate}\item Representação de dados
\item Conceitos e Representação de algoritmos
\item Fluxograma e portugol
\item Conceitos básicos de programação, valores, tipos e expressões
\item Variáveis, comandos de atribuição e de entrada e saída
\item Comandos de controle de fluxo
\item Comando de decisão (if)
\item Comandos de decisão múltipla, de salto (switch, break)
\item Comando de repetição (while, do-while, for)
\item Sub-programação: Funções; procedimentos e parâmetros
\item Estruturas de dados homogêneas (vetores)
\item Cadeia de caracteres (strings)
\item Estruturas de dados homogêneas (Matrizes)
\item Estrutura heterogêneas
\item Apontadores e memória dinâmica (Ponteiros)
\item Arquivos\end{enumerate}}\\
\multicolumn{4}{|p{16cm}|}{}\\
\hline
\multicolumn{4}{|p{16cm}|}{Bibliografia Básica:}\\
\multicolumn{4}{|p{16cm}|}{%
\begin{itemize}\item DEITEL, P.; DEITEL, H. M. C: como programar. 6. ed. São Paulo: Pearson Education, 2011.
\item DEITEL, H. M.; DEITEL, P. J. C++: como programar. 5. ed. São Paulo: Pearson Prentice Hall, 2006.
\item SOUZA, M. A. F. de. Algoritmos e lógica de programação. São Paulo: Cengage Learning, 2005.\end{itemize}}\\
\multicolumn{4}{|p{16cm}|}{}\\
\hline
\multicolumn{4}{|p{16cm}|}{Bibliografia Complementar:}\\
\multicolumn{4}{|p{16cm}|}{%
\begin{itemize}\item ASCENCIO, A. F. G.; CAMPOS, E. A. V. Fundamentos da programação de computadores: algoritmos, pascal e c/c++. 2. ed. São Paulo: Prentice-Hall, 2007.
\item GUEDES, S. Lógica de Programação Algorítmica. São Paulo: Pearson Education do Brasil, 2014.
\item MIZRAHI, V. V. Treinamento em linguagem C. 2. ed. São Paulo: Pearson Education, 2010.
\item MIZRAHI, V. V. Treinamento em linguagem C++: módulo 2. São Paulo: Pearson Education, 2006.
\item SAVITCH, W. J. C++ absoluto. São Paulo: Pearson Education: Addison Wesley, 2004.\end{itemize}}\\
\hline
\end{longtable}
\end{center}

\clearpage
\end{document}