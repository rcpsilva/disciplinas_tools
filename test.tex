\documentclass[11pt]{article}
\usepackage[utf8]{inputenc}
\usepackage[T1]{fontenc}
\usepackage[brazil]{babel}
\usepackage{geometry}
\usepackage{longtable} 
\usepackage{graphicx}
\usepackage{multirow}
\geometry{a4paper,margin=2cm}
\usepackage{enumitem,tabularx,setspace,ragged2e}

\newcommand{\chk}[1]{\ifx#1X\setlength\fboxsep{1pt}\fbox{\rule{1.2ex}{0pt}X}%
\else\setlength\fboxsep{1pt}\fbox{\rule{1.2ex}{0pt}\rule{1.2ex}{0pt}}\fi}

\setlist{noitemsep,leftmargin=*}
\renewcommand\arraystretch{1.15}

\begin{document}
\pagestyle{empty}

\begin{center}
\begin{tabular}{lccr}
 \multirow{3}{*}{\includegraphics[height=2.7cm]{brasao.png}} & \multicolumn{2}{c}{\bfseries UNIVERSIDADE FEDERAL DE OURO PRETO} & \ \ \ \ \ \multirow{3}{*}{\includegraphics[height=2.7cm]{ufop.png}} \\
 & \multicolumn{2}{c}{\bfseries PRÓ-REITORIA DE GRADUAÇÃO} & \\
 & \multicolumn{2}{c}{} & \\
 & \multicolumn{2}{c}{\Large\bfseries PROGRAMA DE DISCIPLINA} & \\
\end{tabular}
\end{center}

\begin{center}
\begin{longtable}{|p{4cm}|p{4cm}|p{4cm}|p{4cm}|}  % repeats {c|} 18 times
\hline
\multicolumn{3}{|p{12cm}|}{Nome do Componente Currigcular em Português:} & \multicolumn{1}{p{4cm}|}{Código:} \\ 
\multicolumn{3}{|p{12cm}|}{\textbf{Matemática Discreta I}} &  \textbf{BCC101}\\ 
\multicolumn{3}{|p{12cm}|}{Nome do Componente Currigcular em Inglês:} &  \\ 
\multicolumn{3}{|p{12cm}|}{\textbf{Discrete Mathematics I}} &  \\ 
\hline
\multicolumn{3}{|p{12cm}|}{Nome e Sigla do Departamento} & Unidade Acadêmica: \\ 
\multicolumn{3}{|p{12cm}|}{Departamento de Computação (DECOM)} & ICEB \\ 
\hline
\multicolumn{4}{|p{12cm}|}{Modalidade de Oferta: [x] presencial \hspace{1cm} [\ ] à distância}\\
\multicolumn{4}{|p{12cm}|}{}\\
\hline
\multicolumn{2}{|p{8cm}|}{Carga horária semestral} & \multicolumn{2}{p{8cm}|}{Carga horária semanal}\\
\multicolumn{2}{|p{8cm}|}{} & \multicolumn{2}{p{8cm}|}{}\\
\hline
\multicolumn{1}{|p{4cm}|}{Total} & \multicolumn{1}{p{4cm}|}{Extencionista} &  \multicolumn{1}{p{4cm}|}{Teórica} &  \multicolumn{1}{p{4cm}|}{Prática} \\
\multicolumn{1}{|p{4cm}|}{60 horas} & \multicolumn{1}{p{4cm}|}{0 horas} &  \multicolumn{1}{p{4cm}|}{04 horas/aula} &  \multicolumn{1}{p{4cm}|}{00 horas/aula} \\
\hline
\multicolumn{4}{|p{16cm}|}{Ementa:}\\
\multicolumn{4}{|p{16cm}|}{Introdução à teoria de conjuntos: definições de conjuntos, operações sobre conjuntos, cardinalidade de conjuntos. Funções: conceitos básicos, composição, funções recursivas. Lógica proposicional e lógica de predicados: sintaxe, semântica e sistema de dedução. Estratégias de prova. Indução e recursão.}\\
\multicolumn{4}{|p{16cm}|}{}\\
\hline
\multicolumn{4}{|p{16cm}|}{Conteúdo programático:}\\
\multicolumn{4}{|p{16cm}|}{\begin{enumerate}
\item Introdução e Revisão de Teoria de Conjuntos
\item Sintaxe e Semântica da Lógica Proposicional
\item Sistema de Dedução da Lógica Proposicional
\item Álgebra Booleana
\item Sintaxe e Semântica da Lógica de Predicados
\item Sistema de Dedução - Lógica de Predicados
\item Álgebra de Predicados
\item Estratégias de prova
\item Indução e Recursão
\item Provas e correção de provas
\end{enumerate}}\\
\multicolumn{4}{|p{16cm}|}{}\\
\hline
\multicolumn{4}{|p{16cm}|}{Bibliografia Básica:}\\
\multicolumn{4}{|p{16cm}|}{\begin{itemize}
\item VELLEMAN, Daniel J. How to Prove It: A Structured Approach. Cambridge: Cambridge University Press, 2006.
\item ROSEN, Kenneth H. Matemática Discreta e suas Aplicações. 6. ed. São Paulo: McGraw-Hill, 2009.
\item O´DONNELL, John; HALL, Cordelia; PAGE, Rex. Discrete Mathematics Using a Computer. Glasgow: Springer-Verlag, 2000.
\end{itemize}}\\
\multicolumn{4}{|p{16cm}|}{} \\
\hline
\multicolumn{4}{|p{16cm}|}{Bibliografia Complementar:}\\
\multicolumn{4}{|p{16cm}|}{\begin{itemize}
\item HUTH, Michael; RYAN, Mark. Lógica em Ciência da Computação: Modelagem e Argumentação sobre Sistemas. 2. ed. Rio de Janeiro: LTC, 2008.
\item SCHEINERMAN, Edward R. Matemática Discreta: Uma Introdução. São Paulo: Cengage Learning, 2011.
\item GERSTING, Judith L. Fundamentos Matemáticos para a Ciência da Computação. 5. ed. Rio de Janeiro: LTC, 2004.
\end{itemize}}\\
\multicolumn{4}{|p{16cm}|}{} \\
\hline
\end{longtable}
\end{center}


\end{document}
