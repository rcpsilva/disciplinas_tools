\documentclass[11pt]{article}
\usepackage{ifthen}
\usepackage[utf8]{inputenc}
\usepackage[T1]{fontenc}
\usepackage[brazil]{babel}
\usepackage{geometry}
\geometry{a4paper,margin=2cm}
\usepackage{longtable,graphicx,multirow,enumitem,tabularx,setspace,ragged2e}
\setlist{noitemsep,leftmargin=*}
\renewcommand\arraystretch{1.15}
% Caixa de seleção
\newcommand{\chk}[1]{\ifx#1X\setlength\fboxsep{1pt}\fbox{\rule{1.3ex}{0pt}X}%
\else\setlength\fboxsep{1pt}\fbox{\rule{1.3ex}{0pt}\rule{1.3ex}{0pt}}\fi}
\pagestyle{empty}
\begin{document}
\begin{center}
\begin{tabular}{lccr}
 \multirow{3}{*}{\includegraphics[height=2.7cm]{brasao.png}} &
 \multicolumn{2}{c}{\bfseries UNIVERSIDADE FEDERAL DE OURO PRETO} &
 \ \ \ \ \multirow{3}{*}{\includegraphics[height=2.7cm]{ufop.png}} \\
 & \multicolumn{2}{c}{\bfseries PRÓ-REITORIA DE GRADUAÇÃO} & \\
 & \multicolumn{2}{c}{} & \\
 & \multicolumn{2}{c}{\Large\bfseries PROGRAMA DE DISCIPLINA} & \\
\end{tabular}
\end{center}

\begin{center}
\begin{longtable}{|p{4cm}|p{4cm}|p{4cm}|p{4cm}|}
\hline
\multicolumn{3}{|p{12cm}|}{Nome do Componente Curricular em Português:} &
\multicolumn{1}{p{4cm}|}{Código:} \\ 
\multicolumn{3}{|p{12cm}|}{\textbf{Arquitetura e Organização de Computadores}} &
\textbf{BIA001}\\ 
\multicolumn{3}{|p{12cm}|}{Nome do Componente Curricular em Inglês:} & \\ 
\multicolumn{3}{|p{12cm}|}{\textbf{Computer Architecture and Organization}} & \\ 
\hline
\multicolumn{3}{|p{12cm}|}{Nome e Sigla do Departamento} & Unidade Acadêmica: \\ 
\multicolumn{3}{|p{12cm}|}{Departamento de Computação (DECOM)} & {ICEB} \\ 
\hline
\multicolumn{4}{|p{16cm}|}{Modalidade de Oferta:
[X] presencial \hspace{1cm}
[ ] à distância}\\
\hline
\multicolumn{2}{|p{8cm}|}{Carga horária semestral} &
\multicolumn{2}{p{8cm}|}{Carga horária semanal}\\
\hline
\multicolumn{1}{|p{4cm}|}{Total} &
\multicolumn{1}{p{4cm}|}{Extensionista} &
\multicolumn{1}{p{4cm}|}{Teórica} &
\multicolumn{1}{p{4cm}|}{Prática} \\ 
\multicolumn{1}{|p{4cm}|}{60\,horas} &
\multicolumn{1}{p{4cm}|}{0\;horas} &
\multicolumn{1}{p{4cm}|}{4\;horas/aula} &
\multicolumn{1}{p{4cm}|}{0\;horas/aula} \\ 
\hline
\multicolumn{4}{|p{16cm}|}{Ementa:}\\
\multicolumn{4}{|p{16cm}|}{}\\
\multicolumn{4}{|p{16cm}|}{Estudo dos componentes básicos de um computador e suas interações. Conceitos de organização e arquitetura de computadores. Sistemas de numeração e codificação. Conjuntos de instruções, ciclo de execução, hierarquia de memória e barramentos.}\\
\multicolumn{4}{|p{16cm}|}{}\\
\hline
\multicolumn{4}{|p{16cm}|}{Conteúdo programático:}\\
\multicolumn{4}{|p{16cm}|}{%
\begin{enumerate}\item Conceitos fundamentais: diferença entre organização e arquitetura de computadores.
\item Sistemas de numeração: binário, octal, decimal e hexadecimal.
\item Aritmética computacional: operações com números inteiros e de ponto flutuante.
\item Unidade Central de Processamento (UCP): registradores, ALU, UC.
\item Ciclo de instrução: busca, decodificação e execução.
\item Conjuntos de instruções (ISA): instruções RISC e CISC.
\item Hierarquia de memória: registradores, cache, RAM, memória secundária.
\item Sistemas de entrada e saída: interfaces e dispositivos.
\item Barramentos e comunicação entre componentes.
\item Noções de pipelines e paralelismo em processadores modernos.\end{enumerate}}\\
\multicolumn{4}{|p{16cm}|}{}\\
\multicolumn{4}{|p{16cm}|}{}\\
\multicolumn{4}{|p{16cm}|}{\vspace{-1cm}}\\
\multicolumn{4}{|p{16cm}|}{}\\
\hline
\multicolumn{4}{|p{16cm}|}{Bibliografia Básica:}\\
\multicolumn{4}{|p{16cm}|}{%
\begin{itemize}\item STALLINGS, William. Arquitetura e organização de computadores. 10. ed. São Paulo: Pearson, 2016. E-book. Disponível em: https://plataforma.bvirtual.com.br. Acesso em: jun/2025.
\item MAK, Ronald. Organização de computadores: arquitetura, interfaces e sistemas operacionais. 1. ed. São Paulo: Cengage Learning, 2020. E-book. Disponível em: https://plataforma.bvirtual.com.br. Acesso em: jun/2025.
\item TANENBAUM, Andrew S.; AUSTIN, Todd. Estruturas de computadores. 6. ed. São Paulo: Pearson, 2014. E-book. Disponível em: https://plataforma.bvirtual.com.br. Acesso em: jun/2025.\end{itemize}}\\
\multicolumn{4}{|p{16cm}|}{}\\
\hline
\multicolumn{4}{|p{16cm}|}{Bibliografia Complementar:}\\
\multicolumn{4}{|p{16cm}|}{%
\begin{itemize}\item BRITO, Fábio de Souza. Arquitetura e organização de computadores: teoria e prática. 1. ed. São Paulo: Érica, 2022. E-book. Disponível em: https://plataforma.bvirtual.com.br. Acesso em: jun/2025.
\item HENNESSY, John L.; PATTERSON, David A. Organização e projeto de computadores: a interface hardware/software. 5. ed. Porto Alegre: Bookman, 2014. E-book. Disponível em: https://plataforma.bvirtual.com.br. Acesso em: jun/2025.
\item SANTOS, Silas P. Fundamentos de organização de computadores. 2. ed. Curitiba: Intersaberes, 2021. E-book. Disponível em: https://plataforma.bvirtual.com.br. Acesso em: jun/2025.
\item MACHADO, Fábio; MAIA, Marcos. Arquitetura de computadores moderna. 1. ed. Rio de Janeiro: LTC, 2019. E-book. Disponível em: https://plataforma.bvirtual.com.br. Acesso em: jun/2025.
\item FERREIRA, Luiz. Sistemas digitais e arquitetura de computadores. 3. ed. São Paulo: Érica, 2018. E-book. Disponível em: https://plataforma.bvirtual.com.br. Acesso em: jun/2025.\end{itemize}}\\
\hline
\end{longtable}
\end{center}

\clearpage
\begin{center}
\begin{tabular}{lccr}
 \multirow{3}{*}{\includegraphics[height=2.7cm]{brasao.png}} &
 \multicolumn{2}{c}{\bfseries UNIVERSIDADE FEDERAL DE OURO PRETO} &
 \ \ \ \ \multirow{3}{*}{\includegraphics[height=2.7cm]{ufop.png}} \\
 & \multicolumn{2}{c}{\bfseries PRÓ-REITORIA DE GRADUAÇÃO} & \\
 & \multicolumn{2}{c}{} & \\
 & \multicolumn{2}{c}{\Large\bfseries PROGRAMA DE DISCIPLINA} & \\
\end{tabular}
\end{center}

\begin{center}
\begin{longtable}{|p{4cm}|p{4cm}|p{4cm}|p{4cm}|}
\hline
\multicolumn{3}{|p{12cm}|}{Nome do Componente Curricular em Português:} &
\multicolumn{1}{p{4cm}|}{Código:} \\ 
\multicolumn{3}{|p{12cm}|}{\textbf{Introdução à Inteligência Artificial}} &
\textbf{BIA002}\\ 
\multicolumn{3}{|p{12cm}|}{Nome do Componente Curricular em Inglês:} & \\ 
\multicolumn{3}{|p{12cm}|}{\textbf{Introduction to Artificial Intelligence}} & \\ 
\hline
\multicolumn{3}{|p{12cm}|}{Nome e Sigla do Departamento} & Unidade Acadêmica: \\ 
\multicolumn{3}{|p{12cm}|}{Departamento de Computação (DECOM)} & {ICEB} \\ 
\hline
\multicolumn{4}{|p{16cm}|}{Modalidade de Oferta:
[X] presencial \hspace{1cm}
[ ] à distância}\\
\hline
\multicolumn{2}{|p{8cm}|}{Carga horária semestral} &
\multicolumn{2}{p{8cm}|}{Carga horária semanal}\\
\hline
\multicolumn{1}{|p{4cm}|}{Total} &
\multicolumn{1}{p{4cm}|}{Extensionista} &
\multicolumn{1}{p{4cm}|}{Teórica} &
\multicolumn{1}{p{4cm}|}{Prática} \\ 
\multicolumn{1}{|p{4cm}|}{60\,horas} &
\multicolumn{1}{p{4cm}|}{0\;horas} &
\multicolumn{1}{p{4cm}|}{4\;horas/aula} &
\multicolumn{1}{p{4cm}|}{0\;horas/aula} \\ 
\hline
\multicolumn{4}{|p{16cm}|}{Ementa:}\\
\multicolumn{4}{|p{16cm}|}{}\\
\multicolumn{4}{|p{16cm}|}{O que é Inteligência Artificial; o comportamento do aluno de Inteligência Artificial; áreas de pesquisa do Departamento de Computação; áreas nas quais atuam os profissionais em Inteligência Artificial.}\\
\multicolumn{4}{|p{16cm}|}{}\\
\hline
\multicolumn{4}{|p{16cm}|}{Conteúdo programático:}\\
\multicolumn{4}{|p{16cm}|}{%
\begin{enumerate}\item A área de Inteligência Artificial e suas oportunidades atuais.
\item Áreas de atuação e mercado de trabalho em Inteligência Artificial.
\item O curso de Inteligência Artificial: grade curricular, áreas, relações entre as disciplinas.
\item Organização política da universidade e institutos/unidades.
\item Representação acadêmica: centro acadêmico e movimentos estudantis.
\item Seminários sobre as áreas de pesquisa e extensão dos professores do DECOM.
\item Apresentação das atividades dos laboratórios de pesquisa e extensão.
\item O comportamento do aluno de Inteligência Artificial: organização e conselhos para estudar melhor e ter sucesso no curso.
\item Seminários de profissionais sobre o mercado de trabalho e carreiras na área de Inteligência Artificial.\end{enumerate}}\\
\multicolumn{4}{|p{16cm}|}{}\\
\multicolumn{4}{|p{16cm}|}{}\\
\multicolumn{4}{|p{16cm}|}{\vspace{-1cm}}\\
\multicolumn{4}{|p{16cm}|}{}\\
\hline
\multicolumn{4}{|p{16cm}|}{Bibliografia Básica:}\\
\multicolumn{4}{|p{16cm}|}{%
\begin{itemize}\item MORAIS, Regis de (org.). Filosofia da ciência e da tecnologia: introdução metodológica e crítica. 1. ed. Campinas: Papirus, 2013. E-book. Disponível em: https://plataforma.bvirtual.com.br. Acesso em: jun/2025.
\item MEDEIROS, Luciano Frontino de. Inteligência artificial aplicada: uma abordagem introdutória. Curitiba, PR: Intersaberes, 2018. E-book. Disponível em: https://plataforma.bvirtual.com.br. Acesso em: jun/2025.
\item DIAS, Ana Francisca Pinto et al.; GUIMARÃES, João Alexandre Silva Alves; ALVES, Rodrigo Vitorino Souza (org.). Os direitos humanos e a ética na era da inteligência artificial. Indaiatuba, SP: Foco, 2023. E-book. Disponível em: https://plataforma.bvirtual.com.br. Acesso em: jun/2025.\end{itemize}}\\
\multicolumn{4}{|p{16cm}|}{}\\
\hline
\multicolumn{4}{|p{16cm}|}{Bibliografia Complementar:}\\
\multicolumn{4}{|p{16cm}|}{%
\begin{itemize}\item KRELLING NETO, Antonio Osmar. Responsabilidade civil: cibercrimes. 1. ed. São Paulo: Contentus, 2020. E-book. Disponível em: https://plataforma.bvirtual.com.br. Acesso em: jun/2025.
\item TAURION, Cezar. Big data. 1. ed. Rio de Janeiro: Brasport, 2013. E-book. Disponível em: https://plataforma.bvirtual.com.br. Acesso em: jun/2025.
\item MUNHOZ, Antonio Siemsen. Responsabilidade e autoridade social das empresas. 1. ed. Curitiba: Intersaberes, 2015. E-book. Disponível em: https://plataforma.bvirtual.com.br. Acesso em: jun/2025.
\item FLORES, Márcio José das; BESS, Alexandre Leal. Inteligência artificial aplicada a negócios. Curitiba, PR: Intersaberes, 2023. E-book. Disponível em: https://plataforma.bvirtual.com.br. Acesso em: jun/2025.
\item MUNIZ, Antonio et al. Inteligência artificial: entenda como a IA pode impactar no mercado de trabalho e na sociedade. [S.l.]: Brasport, 2024. E-book. Disponível em: https://plataforma.bvirtual.com.br. Acesso em: jun/2025.\end{itemize}}\\
\hline
\end{longtable}
\end{center}

\clearpage
\begin{center}
\begin{tabular}{lccr}
 \multirow{3}{*}{\includegraphics[height=2.7cm]{brasao.png}} &
 \multicolumn{2}{c}{\bfseries UNIVERSIDADE FEDERAL DE OURO PRETO} &
 \ \ \ \ \multirow{3}{*}{\includegraphics[height=2.7cm]{ufop.png}} \\
 & \multicolumn{2}{c}{\bfseries PRÓ-REITORIA DE GRADUAÇÃO} & \\
 & \multicolumn{2}{c}{} & \\
 & \multicolumn{2}{c}{\Large\bfseries PROGRAMA DE DISCIPLINA} & \\
\end{tabular}
\end{center}

\begin{center}
\begin{longtable}{|p{4cm}|p{4cm}|p{4cm}|p{4cm}|}
\hline
\multicolumn{3}{|p{12cm}|}{Nome do Componente Curricular em Português:} &
\multicolumn{1}{p{4cm}|}{Código:} \\ 
\multicolumn{3}{|p{12cm}|}{\textbf{Fundamentos de Inteligência Artificial}} &
\textbf{BIA003}\\ 
\multicolumn{3}{|p{12cm}|}{Nome do Componente Curricular em Inglês:} & \\ 
\multicolumn{3}{|p{12cm}|}{\textbf{Foundations of Artificial Intelligence}} & \\ 
\hline
\multicolumn{3}{|p{12cm}|}{Nome e Sigla do Departamento} & Unidade Acadêmica: \\ 
\multicolumn{3}{|p{12cm}|}{Departamento de Computação (DECOM)} & {ICEB} \\ 
\hline
\multicolumn{4}{|p{16cm}|}{Modalidade de Oferta:
[X] presencial \hspace{1cm}
[ ] à distância}\\
\hline
\multicolumn{2}{|p{8cm}|}{Carga horária semestral} &
\multicolumn{2}{p{8cm}|}{Carga horária semanal}\\
\hline
\multicolumn{1}{|p{4cm}|}{Total} &
\multicolumn{1}{p{4cm}|}{Extensionista} &
\multicolumn{1}{p{4cm}|}{Teórica} &
\multicolumn{1}{p{4cm}|}{Prática} \\ 
\multicolumn{1}{|p{4cm}|}{60\,horas} &
\multicolumn{1}{p{4cm}|}{0\;horas} &
\multicolumn{1}{p{4cm}|}{4\;horas/aula} &
\multicolumn{1}{p{4cm}|}{0\;horas/aula} \\ 
\hline
\multicolumn{4}{|p{16cm}|}{Ementa:}\\
\multicolumn{4}{|p{16cm}|}{}\\
\multicolumn{4}{|p{16cm}|}{Fundamentos históricos e conceituais da Inteligência Artificial. Representação do conhecimento. Raciocínio lógico e simbólico. Técnicas de busca. Planejamento. Introdução à incerteza e raciocínio probabilístico.}\\
\multicolumn{4}{|p{16cm}|}{}\\
\hline
\multicolumn{4}{|p{16cm}|}{Conteúdo programático:}\\
\multicolumn{4}{|p{16cm}|}{%
\begin{enumerate}\item História e definições da Inteligência Artificial.
\item Raciocínio simbólico versus conexionista.
\item Agentes inteligentes: definição, arquitetura e ambientes.
\item Representação de conhecimento: lógica proposicional e lógica de predicados.
\item Técnicas de busca: busca não-informada (em profundidade, em largura), busca heurística (A*, gulosa).
\item Espaços de estados e problemas clássicos (jogos, labirintos, quebra-cabeças).
\item Planejamento: introdução ao planejamento de ações.
\item Raciocínio sob incerteza: introdução à probabilidade em IA.
\item Redes Bayesianas: conceitos iniciais e aplicações.
\item Ética e implicações sociais dos sistemas inteligentes.\end{enumerate}}\\
\multicolumn{4}{|p{16cm}|}{}\\
\multicolumn{4}{|p{16cm}|}{}\\
\multicolumn{4}{|p{16cm}|}{\vspace{-1cm}}\\
\multicolumn{4}{|p{16cm}|}{}\\
\hline
\multicolumn{4}{|p{16cm}|}{Bibliografia Básica:}\\
\multicolumn{4}{|p{16cm}|}{%
\begin{itemize}\item RUSSELL, Stuart; NORVIG, Peter. Inteligência artificial. 3. ed. São Paulo: Pearson, 2013. E-book. Disponível em: https://plataforma.bvirtual.com.br. Acesso em: jun/2025.
\item MEDEIROS, Luciano Frontino de. Inteligência artificial aplicada: uma abordagem introdutória. Curitiba, PR: Intersaberes, 2018. E-book. Disponível em: https://plataforma.bvirtual.com.br. Acesso em: jun/2025.
\item BRAGA, A. P.; CARVALHO, A. C. P. de L. F. Redes neurais artificiais: teoria e aplicações. 2. ed. Rio de Janeiro: LTC, 2011. E-book. Disponível em: https://plataforma.bvirtual.com.br. Acesso em: jun/2025.\end{itemize}}\\
\multicolumn{4}{|p{16cm}|}{}\\
\hline
\multicolumn{4}{|p{16cm}|}{Bibliografia Complementar:}\\
\multicolumn{4}{|p{16cm}|}{%
\begin{itemize}\item DIAS, Ana Francisca Pinto et al. (org.). Os direitos humanos e a ética na era da inteligência artificial. Indaiatuba, SP: Foco, 2023. E-book. Disponível em: https://plataforma.bvirtual.com.br. Acesso em: jun/2025.
\item LUGER, George F. Inteligência artificial: estruturas e estratégias para a solução de problemas complexos. 6. ed. Porto Alegre: Bookman, 2004. E-book. Disponível em: https://plataforma.bvirtual.com.br. Acesso em: jun/2025.
\item MUNIZ, Antonio et al. Inteligência artificial: entenda como a IA pode impactar no mercado de trabalho e na sociedade. [S.l.]: Brasport, 2024. E-book. Disponível em: https://plataforma.bvirtual.com.br. Acesso em: jun/2025.
\item GOODFELLOW, Ian; BENGIO, Yoshua; COURVILLE, Aaron. Deep learning. Cambridge: MIT Press, 2016. (Uso complementar para contextualização moderna, capítulos introdutórios).
\item BARROS, L. N. de. Introdução à inteligência artificial. 1. ed. Rio de Janeiro: LTC, 2010. E-book. Disponível em: https://plataforma.bvirtual.com.br. Acesso em: jun/2025.\end{itemize}}\\
\hline
\end{longtable}
\end{center}

\clearpage
\begin{center}
\begin{tabular}{lccr}
 \multirow{3}{*}{\includegraphics[height=2.7cm]{brasao.png}} &
 \multicolumn{2}{c}{\bfseries UNIVERSIDADE FEDERAL DE OURO PRETO} &
 \ \ \ \ \multirow{3}{*}{\includegraphics[height=2.7cm]{ufop.png}} \\
 & \multicolumn{2}{c}{\bfseries PRÓ-REITORIA DE GRADUAÇÃO} & \\
 & \multicolumn{2}{c}{} & \\
 & \multicolumn{2}{c}{\Large\bfseries PROGRAMA DE DISCIPLINA} & \\
\end{tabular}
\end{center}

\begin{center}
\begin{longtable}{|p{4cm}|p{4cm}|p{4cm}|p{4cm}|}
\hline
\multicolumn{3}{|p{12cm}|}{Nome do Componente Curricular em Português:} &
\multicolumn{1}{p{4cm}|}{Código:} \\ 
\multicolumn{3}{|p{12cm}|}{\textbf{Inteligência Artificial Clássica}} &
\textbf{BIA004}\\ 
\multicolumn{3}{|p{12cm}|}{Nome do Componente Curricular em Inglês:} & \\ 
\multicolumn{3}{|p{12cm}|}{\textbf{Classical Artificial Intelligence}} & \\ 
\hline
\multicolumn{3}{|p{12cm}|}{Nome e Sigla do Departamento} & Unidade Acadêmica: \\ 
\multicolumn{3}{|p{12cm}|}{Departamento de Computação (DECOM)} & {ICEB} \\ 
\hline
\multicolumn{4}{|p{16cm}|}{Modalidade de Oferta:
[X] presencial \hspace{1cm}
[ ] à distância}\\
\hline
\multicolumn{2}{|p{8cm}|}{Carga horária semestral} &
\multicolumn{2}{p{8cm}|}{Carga horária semanal}\\
\hline
\multicolumn{1}{|p{4cm}|}{Total} &
\multicolumn{1}{p{4cm}|}{Extensionista} &
\multicolumn{1}{p{4cm}|}{Teórica} &
\multicolumn{1}{p{4cm}|}{Prática} \\ 
\multicolumn{1}{|p{4cm}|}{60\,horas} &
\multicolumn{1}{p{4cm}|}{0\;horas} &
\multicolumn{1}{p{4cm}|}{4\;horas/aula} &
\multicolumn{1}{p{4cm}|}{0\;horas/aula} \\ 
\hline
\multicolumn{4}{|p{16cm}|}{Ementa:}\\
\multicolumn{4}{|p{16cm}|}{}\\
\multicolumn{4}{|p{16cm}|}{Estudo das abordagens simbólicas da Inteligência Artificial. Técnicas de busca, raciocínio lógico, planejamento, representação de conhecimento e sistemas especialistas. Aplicações e limitações da IA clássica.}\\
\multicolumn{4}{|p{16cm}|}{}\\
\hline
\multicolumn{4}{|p{16cm}|}{Conteúdo programático:}\\
\multicolumn{4}{|p{16cm}|}{%
\begin{enumerate}\item Conceitos e motivações da IA clássica: simbolismo e raciocínio deliberativo.
\item Modelagem de agentes racionais baseados em símbolos.
\item Espaços de estados e resolução de problemas com busca.
\item Busca não informada: em largura, em profundidade, custo uniforme.
\item Busca informada: A*, busca gulosa e heurísticas.
\item Representação de conhecimento: lógica proposicional e lógica de predicados de primeira ordem.
\item Inferência lógica: resolução, unificação e encadeamento (forward/backward chaining).
\item Planejamento clássico: STRIPS, espaço de planos e busca em espaço de estados.
\item Sistemas especialistas: arquitetura, motor de inferência, base de regras.
\item Aplicações clássicas: jogos, robótica simbólica, agentes baseados em conhecimento.\end{enumerate}}\\
\multicolumn{4}{|p{16cm}|}{}\\
\multicolumn{4}{|p{16cm}|}{}\\
\multicolumn{4}{|p{16cm}|}{\vspace{-1cm}}\\
\multicolumn{4}{|p{16cm}|}{}\\
\hline
\multicolumn{4}{|p{16cm}|}{Bibliografia Básica:}\\
\multicolumn{4}{|p{16cm}|}{%
\begin{itemize}\item RUSSELL, Stuart; NORVIG, Peter. Inteligência artificial. 3. ed. São Paulo: Pearson, 2013. E-book. Disponível em: https://plataforma.bvirtual.com.br. Acesso em: jun/2025.
\item LUGER, George F. Inteligência artificial: estruturas e estratégias para a solução de problemas complexos. 6. ed. Porto Alegre: Bookman, 2004. E-book. Disponível em: https://plataforma.bvirtual.com.br. Acesso em: jun/2025.
\item BARROS, L. N. de. Introdução à inteligência artificial. 1. ed. Rio de Janeiro: LTC, 2010. E-book. Disponível em: https://plataforma.bvirtual.com.br. Acesso em: jun/2025.\end{itemize}}\\
\multicolumn{4}{|p{16cm}|}{}\\
\hline
\multicolumn{4}{|p{16cm}|}{Bibliografia Complementar:}\\
\multicolumn{4}{|p{16cm}|}{%
\begin{itemize}\item MEDEIROS, Luciano Frontino de. Inteligência artificial aplicada: uma abordagem introdutória. Curitiba, PR: Intersaberes, 2018. E-book. Disponível em: https://plataforma.bvirtual.com.br. Acesso em: jun/2025.
\item BARR, Avron; FEIGENBAUM, Edward A. The Handbook of Artificial Intelligence. Volume I. Reading, MA: Addison-Wesley, 1981. (clássico da área; disponível em bibliotecas digitais).
\item NILSSON, Nils J. Artificial Intelligence: A New Synthesis. San Francisco: Morgan Kaufmann, 1998.
\item DIAS, Ana Francisca Pinto et al. (org.). Os direitos humanos e a ética na era da inteligência artificial. Indaiatuba, SP: Foco, 2023. E-book. Disponível em: https://plataforma.bvirtual.com.br. Acesso em: jun/2025.
\item LAUFER, C. Sistemas especialistas e representação do conhecimento. 1. ed. São Paulo: Érica, 2009. E-book. Disponível em: https://plataforma.bvirtual.com.br. Acesso em: jun/2025.\end{itemize}}\\
\hline
\end{longtable}
\end{center}

\clearpage
\begin{center}
\begin{tabular}{lccr}
 \multirow{3}{*}{\includegraphics[height=2.7cm]{brasao.png}} &
 \multicolumn{2}{c}{\bfseries UNIVERSIDADE FEDERAL DE OURO PRETO} &
 \ \ \ \ \multirow{3}{*}{\includegraphics[height=2.7cm]{ufop.png}} \\
 & \multicolumn{2}{c}{\bfseries PRÓ-REITORIA DE GRADUAÇÃO} & \\
 & \multicolumn{2}{c}{} & \\
 & \multicolumn{2}{c}{\Large\bfseries PROGRAMA DE DISCIPLINA} & \\
\end{tabular}
\end{center}

\begin{center}
\begin{longtable}{|p{4cm}|p{4cm}|p{4cm}|p{4cm}|}
\hline
\multicolumn{3}{|p{12cm}|}{Nome do Componente Curricular em Português:} &
\multicolumn{1}{p{4cm}|}{Código:} \\ 
\multicolumn{3}{|p{12cm}|}{\textbf{Introdução à Teoria de Otimização: Cálculo Diferencial a Várias Variáveis}} &
\textbf{BIA005}\\ 
\multicolumn{3}{|p{12cm}|}{Nome do Componente Curricular em Inglês:} & \\ 
\multicolumn{3}{|p{12cm}|}{\textbf{Introduction to Optimization Theory: Multivariable Differential Calculus}} & \\ 
\hline
\multicolumn{3}{|p{12cm}|}{Nome e Sigla do Departamento} & Unidade Acadêmica: \\ 
\multicolumn{3}{|p{12cm}|}{Departamento de Computação (DECOM)} & {ICEB} \\ 
\hline
\multicolumn{4}{|p{16cm}|}{Modalidade de Oferta:
[X] presencial \hspace{1cm}
[ ] à distância}\\
\hline
\multicolumn{2}{|p{8cm}|}{Carga horária semestral} &
\multicolumn{2}{p{8cm}|}{Carga horária semanal}\\
\hline
\multicolumn{1}{|p{4cm}|}{Total} &
\multicolumn{1}{p{4cm}|}{Extensionista} &
\multicolumn{1}{p{4cm}|}{Teórica} &
\multicolumn{1}{p{4cm}|}{Prática} \\ 
\multicolumn{1}{|p{4cm}|}{60\,horas} &
\multicolumn{1}{p{4cm}|}{0\;horas} &
\multicolumn{1}{p{4cm}|}{4\;horas/aula} &
\multicolumn{1}{p{4cm}|}{0\;horas/aula} \\ 
\hline
\multicolumn{4}{|p{16cm}|}{Ementa:}\\
\multicolumn{4}{|p{16cm}|}{}\\
\multicolumn{4}{|p{16cm}|}{Estudo de funções de várias variáveis, vetores, derivadas parciais e gradiente. Otimização com e sem restrições. Aplicações em ciência de dados, inteligência artificial e métodos computacionais. Introdução à descida do gradiente e interpretação geométrica.}\\
\multicolumn{4}{|p{16cm}|}{}\\
\hline
\multicolumn{4}{|p{16cm}|}{Conteúdo programático:}\\
\multicolumn{4}{|p{16cm}|}{%
\begin{enumerate}\item Revisão de vetores, matrizes e funções vetoriais.
\item Funções de várias variáveis: definição, continuidade e diferenciabilidade.
\item Derivadas parciais, gradiente e interpretação geométrica.
\item Direções de crescimento e derivada direcional.
\item Máximos, mínimos e pontos críticos de funções escalares.
\item Otimização sem restrições: condições de primeira e segunda ordem.
\item Multiplicadores de Lagrange e otimização com restrições de igualdade.
\item Introdução à convexidade e suas implicações em otimização.
\item Método da descida do gradiente: conceito, motivação e implementação.
\item Aplicações em computação: ajuste de funções, aprendizado de máquina, redes neurais.\end{enumerate}}\\
\multicolumn{4}{|p{16cm}|}{}\\
\multicolumn{4}{|p{16cm}|}{}\\
\multicolumn{4}{|p{16cm}|}{\vspace{-1cm}}\\
\multicolumn{4}{|p{16cm}|}{}\\
\hline
\multicolumn{4}{|p{16cm}|}{Bibliografia Básica:}\\
\multicolumn{4}{|p{16cm}|}{%
\begin{itemize}\item STEWART, James. Cálculo: volume 2. 8. ed. São Paulo: Cengage Learning, 2016. E-book. Disponível em: https://plataforma.bvirtual.com.br. Acesso em: jun/2025.
\item BOYER, Carl D.; GUTTMAN, Richard C. Cálculo com geometria analítica. 2. ed. São Paulo: Makron Books, 2020. E-book. Disponível em: https://plataforma.bvirtual.com.br. Acesso em: jun/2025.
\item NOCEDAL, Jorge; WRIGHT, Stephen J. Numerical Optimization. 2. ed. New York: Springer, 2006. (Capítulos introdutórios).\end{itemize}}\\
\multicolumn{4}{|p{16cm}|}{}\\
\hline
\multicolumn{4}{|p{16cm}|}{Bibliografia Complementar:}\\
\multicolumn{4}{|p{16cm}|}{%
\begin{itemize}\item GILBERT, Jean-Charles; NOCEDAL, Jorge. Otimização contínua: fundamentos numéricos. Rio de Janeiro: LTC, 2019. E-book. Disponível em: https://plataforma.bvirtual.com.br. Acesso em: jun/2025.
\item BOYD, Stephen; VANDENBERGHE, Lieven. Convex Optimization. Cambridge: Cambridge University Press, 2004. (Capítulos introdutórios; referência clássica em IA e ciência de dados).
\item STRANG, Gilbert. Cálculo. São Paulo: Cengage Learning, 2017. E-book. Disponível em: https://plataforma.bvirtual.com.br. Acesso em: jun/2025.
\item GOODFELLOW, Ian; BENGIO, Yoshua; COURVILLE, Aaron. Deep Learning. Cambridge: MIT Press, 2016. (Capítulo sobre otimização e gradient descent).
\item SILVA, Carlos Henrique da; COSTA, Edson de Oliveira. Cálculo vetorial e aplicações. 1. ed. São Paulo: Pearson, 2014. E-book. Disponível em: https://plataforma.bvirtual.com.br. Acesso em: jun/2025.\end{itemize}}\\
\hline
\end{longtable}
\end{center}

\clearpage
\begin{center}
\begin{tabular}{lccr}
 \multirow{3}{*}{\includegraphics[height=2.7cm]{brasao.png}} &
 \multicolumn{2}{c}{\bfseries UNIVERSIDADE FEDERAL DE OURO PRETO} &
 \ \ \ \ \multirow{3}{*}{\includegraphics[height=2.7cm]{ufop.png}} \\
 & \multicolumn{2}{c}{\bfseries PRÓ-REITORIA DE GRADUAÇÃO} & \\
 & \multicolumn{2}{c}{} & \\
 & \multicolumn{2}{c}{\Large\bfseries PROGRAMA DE DISCIPLINA} & \\
\end{tabular}
\end{center}

\begin{center}
\begin{longtable}{|p{4cm}|p{4cm}|p{4cm}|p{4cm}|}
\hline
\multicolumn{3}{|p{12cm}|}{Nome do Componente Curricular em Português:} &
\multicolumn{1}{p{4cm}|}{Código:} \\ 
\multicolumn{3}{|p{12cm}|}{\textbf{Estatística e Probabilidade para Computação}} &
\textbf{BIA006}\\ 
\multicolumn{3}{|p{12cm}|}{Nome do Componente Curricular em Inglês:} & \\ 
\multicolumn{3}{|p{12cm}|}{\textbf{Statistics and Probability for Computing}} & \\ 
\hline
\multicolumn{3}{|p{12cm}|}{Nome e Sigla do Departamento} & Unidade Acadêmica: \\ 
\multicolumn{3}{|p{12cm}|}{Departamento de Computação (DECOM)} & {ICEB} \\ 
\hline
\multicolumn{4}{|p{16cm}|}{Modalidade de Oferta:
[X] presencial \hspace{1cm}
[ ] à distância}\\
\hline
\multicolumn{2}{|p{8cm}|}{Carga horária semestral} &
\multicolumn{2}{p{8cm}|}{Carga horária semanal}\\
\hline
\multicolumn{1}{|p{4cm}|}{Total} &
\multicolumn{1}{p{4cm}|}{Extensionista} &
\multicolumn{1}{p{4cm}|}{Teórica} &
\multicolumn{1}{p{4cm}|}{Prática} \\ 
\multicolumn{1}{|p{4cm}|}{60\,horas} &
\multicolumn{1}{p{4cm}|}{0\;horas} &
\multicolumn{1}{p{4cm}|}{4\;horas/aula} &
\multicolumn{1}{p{4cm}|}{0\;horas/aula} \\ 
\hline
\multicolumn{4}{|p{16cm}|}{Ementa:}\\
\multicolumn{4}{|p{16cm}|}{}\\
\multicolumn{4}{|p{16cm}|}{Conceitos fundamentais de estatística e probabilidade aplicados à computação. Variáveis aleatórias, distribuições de probabilidade, inferência estatística e simulação. Aplicações em ciência de dados, IA e sistemas computacionais.}\\
\multicolumn{4}{|p{16cm}|}{}\\
\hline
\multicolumn{4}{|p{16cm}|}{Conteúdo programático:}\\
\multicolumn{4}{|p{16cm}|}{%
\begin{enumerate}\item Conceitos básicos de estatística descritiva: medidas de tendência central e dispersão.
\item Visualização e análise de dados: histogramas, boxplots e gráficos de dispersão.
\item Experimentos aleatórios, espaço amostral e eventos.
\item Probabilidade: definição clássica, frequentista e axiomática.
\item Teorema de Bayes e independência de eventos.
\item Variáveis aleatórias discretas e contínuas: funções de probabilidade e densidade.
\item Distribuições fundamentais: binomial, geométrica, Poisson, normal, exponencial.
\item Valor esperado, variância e momentos.
\item Teorema Central do Limite.
\item Inferência estatística: estimação pontual e intervalar, testes de hipótese.
\item Simulação de variáveis aleatórias e Monte Carlo.
\item Aplicações computacionais: modelagem de incerteza, IA probabilística, análise de desempenho de algoritmos.\end{enumerate}}\\
\multicolumn{4}{|p{16cm}|}{}\\
\multicolumn{4}{|p{16cm}|}{}\\
\multicolumn{4}{|p{16cm}|}{\vspace{-1cm}}\\
\multicolumn{4}{|p{16cm}|}{}\\
\hline
\multicolumn{4}{|p{16cm}|}{Bibliografia Básica:}\\
\multicolumn{4}{|p{16cm}|}{%
\begin{itemize}\item MONTGOMERY, Douglas C.; RUNGER, George C. Estatística aplicada e probabilidade para engenheiros. 6. ed. Rio de Janeiro: LTC, 2019. E-book. Disponível em: https://plataforma.bvirtual.com.br. Acesso em: jun/2025.
\item NAVIDI, William. Estatística para engenheiros e cientistas. 4. ed. Porto Alegre: AMGH, 2021. E-book. Disponível em: https://plataforma.bvirtual.com.br. Acesso em: jun/2025.
\item LARSON, Ron; FARBER, Betsy. Estatística aplicada. 7. ed. São Paulo: Pearson, 2015. E-book. Disponível em: https://plataforma.bvirtual.com.br. Acesso em: jun/2025.\end{itemize}}\\
\multicolumn{4}{|p{16cm}|}{}\\
\hline
\multicolumn{4}{|p{16cm}|}{Bibliografia Complementar:}\\
\multicolumn{4}{|p{16cm}|}{%
\begin{itemize}\item WASSERMAN, Larry. All of Statistics: A Concise Course in Statistical Inference. New York: Springer, 2004.
\item GRIMSTED, Robert; SNELL, J. Laurie. Introduction to Probability. 2. ed. American Mathematical Society, 2019. (Disponível online em formato gratuito).
\item RUSSELL, Stuart; NORVIG, Peter. Inteligência artificial. 3. ed. São Paulo: Pearson, 2013. (Capítulo sobre raciocínio probabilístico).
\item OLIVEIRA, Hélio Coury Valgas de. Introdução à estatística para ciência de dados. 1. ed. São Paulo: LTC, 2022. E-book. Disponível em: https://plataforma.bvirtual.com.br. Acesso em: jun/2025.
\item GENTLE, James E. Computational Statistics. New York: Springer, 2009. (Para aprofundamento em aplicações com software).\end{itemize}}\\
\hline
\end{longtable}
\end{center}

\clearpage
\begin{center}
\begin{tabular}{lccr}
 \multirow{3}{*}{\includegraphics[height=2.7cm]{brasao.png}} &
 \multicolumn{2}{c}{\bfseries UNIVERSIDADE FEDERAL DE OURO PRETO} &
 \ \ \ \ \multirow{3}{*}{\includegraphics[height=2.7cm]{ufop.png}} \\
 & \multicolumn{2}{c}{\bfseries PRÓ-REITORIA DE GRADUAÇÃO} & \\
 & \multicolumn{2}{c}{} & \\
 & \multicolumn{2}{c}{\Large\bfseries PROGRAMA DE DISCIPLINA} & \\
\end{tabular}
\end{center}

\begin{center}
\begin{longtable}{|p{4cm}|p{4cm}|p{4cm}|p{4cm}|}
\hline
\multicolumn{3}{|p{12cm}|}{Nome do Componente Curricular em Português:} &
\multicolumn{1}{p{4cm}|}{Código:} \\ 
\multicolumn{3}{|p{12cm}|}{\textbf{Inteligência de Negócios e Dados}} &
\textbf{BIA007}\\ 
\multicolumn{3}{|p{12cm}|}{Nome do Componente Curricular em Inglês:} & \\ 
\multicolumn{3}{|p{12cm}|}{\textbf{Business Intelligence and Data}} & \\ 
\hline
\multicolumn{3}{|p{12cm}|}{Nome e Sigla do Departamento} & Unidade Acadêmica: \\ 
\multicolumn{3}{|p{12cm}|}{Departamento de Computação (DECOM)} & {ICEB} \\ 
\hline
\multicolumn{4}{|p{16cm}|}{Modalidade de Oferta:
[X] presencial \hspace{1cm}
[ ] à distância}\\
\hline
\multicolumn{2}{|p{8cm}|}{Carga horária semestral} &
\multicolumn{2}{p{8cm}|}{Carga horária semanal}\\
\hline
\multicolumn{1}{|p{4cm}|}{Total} &
\multicolumn{1}{p{4cm}|}{Extensionista} &
\multicolumn{1}{p{4cm}|}{Teórica} &
\multicolumn{1}{p{4cm}|}{Prática} \\ 
\multicolumn{1}{|p{4cm}|}{60\,horas} &
\multicolumn{1}{p{4cm}|}{0\;horas} &
\multicolumn{1}{p{4cm}|}{4\;horas/aula} &
\multicolumn{1}{p{4cm}|}{0\;horas/aula} \\ 
\hline
\multicolumn{4}{|p{16cm}|}{Ementa:}\\
\multicolumn{4}{|p{16cm}|}{}\\
\multicolumn{4}{|p{16cm}|}{Conceitos e tecnologias para organização, modelagem e análise de dados em apoio à tomada de decisão. Banco de dados relacionais, não relacionais e multidimensionais. Processos de ETL, modelagem dimensional, OLAP e fundamentos de inteligência de negócios.}\\
\multicolumn{4}{|p{16cm}|}{}\\
\hline
\multicolumn{4}{|p{16cm}|}{Conteúdo programático:}\\
\multicolumn{4}{|p{16cm}|}{%
\begin{enumerate}\item Introdução à inteligência de negócios (BI): conceitos, objetivos e aplicações.
\item Modelagem de dados: modelo relacional, MER e DER.
\item Banco de dados relacionais: SQL, normalização, integridade e consultas.
\item NoSQL e bancos de dados não estruturados: documentos, grafos e chave-valor.
\item Modelagem dimensional: fatos, dimensões, estrelas e flocos de neve.
\item Processos ETL: extração, transformação e carga de dados.
\item Armazenamento de dados: data warehouse e data lake.
\item Consultas analíticas e OLAP: operações roll-up, drill-down, slice e dice.
\item Ferramentas de visualização e dashboards: princípios de design de relatórios.
\item Estudos de caso: uso de dados em decisões empresariais e operacionais.\end{enumerate}}\\
\multicolumn{4}{|p{16cm}|}{}\\
\multicolumn{4}{|p{16cm}|}{}\\
\multicolumn{4}{|p{16cm}|}{\vspace{-1cm}}\\
\multicolumn{4}{|p{16cm}|}{}\\
\hline
\multicolumn{4}{|p{16cm}|}{Bibliografia Básica:}\\
\multicolumn{4}{|p{16cm}|}{%
\begin{itemize}\item INMON, William H. Building the Data Warehouse. 4. ed. Indianapolis: Wiley, 2005.
\item KIMBALL, Ralph; ROSS, Margy. The Data Warehouse Toolkit: The Definitive Guide to Dimensional Modeling. 3. ed. Indianapolis: Wiley, 2013.
\item DATE, C. J. Introdução a sistemas de bancos de dados. 8. ed. Rio de Janeiro: Campus, 2004. E-book. Disponível em: https://plataforma.bvirtual.com.br. Acesso em: jun/2025.\end{itemize}}\\
\multicolumn{4}{|p{16cm}|}{}\\
\hline
\multicolumn{4}{|p{16cm}|}{Bibliografia Complementar:}\\
\multicolumn{4}{|p{16cm}|}{%
\begin{itemize}\item HELMERS, Shawn; LARSON, Baya Dewitt. Fundamentos de BI: inteligência de negócios orientada por dados. 2. ed. São Paulo: Alta Books, 2022. E-book. Disponível em: https://plataforma.bvirtual.com.br. Acesso em: jun/2025.
\item KRUG, Michael. Bancos de dados NoSQL: fundamentos e aplicações. 1. ed. São Paulo: Érica, 2021. E-book. Disponível em: https://plataforma.bvirtual.com.br. Acesso em: jun/2025.
\item SILBERSCHATZ, Abraham; KORTH, Henry F.; SUDARSHAN, S. Sistemas de bancos de dados. 6. ed. São Paulo: Pearson, 2013. E-book. Disponível em: https://plataforma.bvirtual.com.br. Acesso em: jun/2025.
\item PONNAIAH, Paulraj. Data Warehousing Fundamentals. 1. ed. New York: Wiley, 2001.
\item FERREIRA, Rodrigo Siqueira. Banco de dados: teoria e prática. 1. ed. São Paulo: Novatec, 2020. E-book. Disponível em: https://plataforma.bvirtual.com.br. Acesso em: jun/2025.\end{itemize}}\\
\hline
\end{longtable}
\end{center}

\clearpage
\begin{center}
\begin{tabular}{lccr}
 \multirow{3}{*}{\includegraphics[height=2.7cm]{brasao.png}} &
 \multicolumn{2}{c}{\bfseries UNIVERSIDADE FEDERAL DE OURO PRETO} &
 \ \ \ \ \multirow{3}{*}{\includegraphics[height=2.7cm]{ufop.png}} \\
 & \multicolumn{2}{c}{\bfseries PRÓ-REITORIA DE GRADUAÇÃO} & \\
 & \multicolumn{2}{c}{} & \\
 & \multicolumn{2}{c}{\Large\bfseries PROGRAMA DE DISCIPLINA} & \\
\end{tabular}
\end{center}

\begin{center}
\begin{longtable}{|p{4cm}|p{4cm}|p{4cm}|p{4cm}|}
\hline
\multicolumn{3}{|p{12cm}|}{Nome do Componente Curricular em Português:} &
\multicolumn{1}{p{4cm}|}{Código:} \\ 
\multicolumn{3}{|p{12cm}|}{\textbf{Aprendizado de Máquina Supervisionado}} &
\textbf{BIA008}\\ 
\multicolumn{3}{|p{12cm}|}{Nome do Componente Curricular em Inglês:} & \\ 
\multicolumn{3}{|p{12cm}|}{\textbf{Supervised Machine Learning}} & \\ 
\hline
\multicolumn{3}{|p{12cm}|}{Nome e Sigla do Departamento} & Unidade Acadêmica: \\ 
\multicolumn{3}{|p{12cm}|}{Departamento de Computação (DECOM)} & {ICEB} \\ 
\hline
\multicolumn{4}{|p{16cm}|}{Modalidade de Oferta:
[X] presencial \hspace{1cm}
[ ] à distância}\\
\hline
\multicolumn{2}{|p{8cm}|}{Carga horária semestral} &
\multicolumn{2}{p{8cm}|}{Carga horária semanal}\\
\hline
\multicolumn{1}{|p{4cm}|}{Total} &
\multicolumn{1}{p{4cm}|}{Extensionista} &
\multicolumn{1}{p{4cm}|}{Teórica} &
\multicolumn{1}{p{4cm}|}{Prática} \\ 
\multicolumn{1}{|p{4cm}|}{60\,horas} &
\multicolumn{1}{p{4cm}|}{0\;horas} &
\multicolumn{1}{p{4cm}|}{4\;horas/aula} &
\multicolumn{1}{p{4cm}|}{0\;horas/aula} \\ 
\hline
\multicolumn{4}{|p{16cm}|}{Ementa:}\\
\multicolumn{4}{|p{16cm}|}{}\\
\multicolumn{4}{|p{16cm}|}{Conceitos e algoritmos de aprendizado de máquina supervisionado. Classificação, regressão, avaliação de modelos e análise de desempenho. Aplicações em ciência de dados e inteligência artificial. Enfoque prático com experimentação computacional.}\\
\multicolumn{4}{|p{16cm}|}{}\\
\hline
\multicolumn{4}{|p{16cm}|}{Conteúdo programático:}\\
\multicolumn{4}{|p{16cm}|}{%
\begin{enumerate}\item Introdução ao aprendizado de máquina: conceitos, tarefas e tipos de aprendizado.
\item Conjuntos de dados supervisionados: características, rótulos e pré-processamento.
\item Conceito de função alvo, hipótese e generalização.
\item Modelos de classificação: k-NN, Naive Bayes, árvores de decisão, regressão logística, SVM.
\item Modelos de regressão: regressão linear simples e múltipla, regularização (Ridge, Lasso).
\item Técnicas de ensemble: Random Forest, Gradient Boosting, bagging e boosting.
\item Divisão de dados: treino, validação e teste; validação cruzada.
\item Métricas de avaliação: acurácia, precisão, recall, F1, curva ROC e AUC.
\item Overfitting e underfitting: diagnóstico e controle (bias-variance tradeoff).
\item Uso de bibliotecas em Python (scikit-learn, pandas, matplotlib) para modelagem supervisionada.\end{enumerate}}\\
\multicolumn{4}{|p{16cm}|}{}\\
\multicolumn{4}{|p{16cm}|}{}\\
\multicolumn{4}{|p{16cm}|}{\vspace{-1cm}}\\
\multicolumn{4}{|p{16cm}|}{}\\
\hline
\multicolumn{4}{|p{16cm}|}{Bibliografia Básica:}\\
\multicolumn{4}{|p{16cm}|}{%
\begin{itemize}\item GERON, Aurélien. Mãos à obra: aprendizado de máquina com Scikit-Learn, Keras e TensorFlow. 2. ed. São Paulo: Novatec, 2020. E-book. Disponível em: https://plataforma.bvirtual.com.br. Acesso em: jun/2025.
\item ALPAYDIN, Ethem. Introduction to Machine Learning. 4. ed. Cambridge: MIT Press, 2020.
\item JAMES, Gareth et al. An Introduction to Statistical Learning. 2. ed. New York: Springer, 2021. (Disponível gratuitamente em https://www.statlearning.com).\end{itemize}}\\
\multicolumn{4}{|p{16cm}|}{}\\
\hline
\multicolumn{4}{|p{16cm}|}{Bibliografia Complementar:}\\
\multicolumn{4}{|p{16cm}|}{%
\begin{itemize}\item GOODFELLOW, Ian; BENGIO, Yoshua; COURVILLE, Aaron. Deep Learning. Cambridge: MIT Press, 2016. (Capítulos introdutórios sobre aprendizado supervisionado).
\item MÜLLER, Andreas C.; GUIDO, Sarah. Introduction to Machine Learning with Python. 1. ed. O'Reilly, 2016.
\item BISHOP, Christopher M. Pattern Recognition and Machine Learning. New York: Springer, 2006.
\item RUSSELL, Stuart; NORVIG, Peter. Inteligência artificial. 3. ed. São Paulo: Pearson, 2013. (Capítulo sobre aprendizado).
\item ZELIKOVSKY, Alexander; SAFRO, Ilya. Introduction to Machine Learning. New York: Springer, 2022.\end{itemize}}\\
\hline
\end{longtable}
\end{center}

\clearpage
\begin{center}
\begin{tabular}{lccr}
 \multirow{3}{*}{\includegraphics[height=2.7cm]{brasao.png}} &
 \multicolumn{2}{c}{\bfseries UNIVERSIDADE FEDERAL DE OURO PRETO} &
 \ \ \ \ \multirow{3}{*}{\includegraphics[height=2.7cm]{ufop.png}} \\
 & \multicolumn{2}{c}{\bfseries PRÓ-REITORIA DE GRADUAÇÃO} & \\
 & \multicolumn{2}{c}{} & \\
 & \multicolumn{2}{c}{\Large\bfseries PROGRAMA DE DISCIPLINA} & \\
\end{tabular}
\end{center}

\begin{center}
\begin{longtable}{|p{4cm}|p{4cm}|p{4cm}|p{4cm}|}
\hline
\multicolumn{3}{|p{12cm}|}{Nome do Componente Curricular em Português:} &
\multicolumn{1}{p{4cm}|}{Código:} \\ 
\multicolumn{3}{|p{12cm}|}{\textbf{Aprendizado de Máquina Não Supervisionado}} &
\textbf{BIA010}\\ 
\multicolumn{3}{|p{12cm}|}{Nome do Componente Curricular em Inglês:} & \\ 
\multicolumn{3}{|p{12cm}|}{\textbf{Unsupervised Machine Learning}} & \\ 
\hline
\multicolumn{3}{|p{12cm}|}{Nome e Sigla do Departamento} & Unidade Acadêmica: \\ 
\multicolumn{3}{|p{12cm}|}{Departamento de Computação (DECOM)} & {ICEB} \\ 
\hline
\multicolumn{4}{|p{16cm}|}{Modalidade de Oferta:
[X] presencial \hspace{1cm}
[ ] à distância}\\
\hline
\multicolumn{2}{|p{8cm}|}{Carga horária semestral} &
\multicolumn{2}{p{8cm}|}{Carga horária semanal}\\
\hline
\multicolumn{1}{|p{4cm}|}{Total} &
\multicolumn{1}{p{4cm}|}{Extensionista} &
\multicolumn{1}{p{4cm}|}{Teórica} &
\multicolumn{1}{p{4cm}|}{Prática} \\ 
\multicolumn{1}{|p{4cm}|}{60\,horas} &
\multicolumn{1}{p{4cm}|}{0\;horas} &
\multicolumn{1}{p{4cm}|}{4\;horas/aula} &
\multicolumn{1}{p{4cm}|}{0\;horas/aula} \\ 
\hline
\multicolumn{4}{|p{16cm}|}{Ementa:}\\
\multicolumn{4}{|p{16cm}|}{}\\
\multicolumn{4}{|p{16cm}|}{Estudo dos principais métodos de aprendizado de máquina não supervisionado. Técnicas de agrupamento, redução de dimensionalidade, extração de características e detecção de anomalias. Aplicações em ciência de dados, visualização e descoberta de padrões.}\\
\multicolumn{4}{|p{16cm}|}{}\\
\hline
\multicolumn{4}{|p{16cm}|}{Conteúdo programático:}\\
\multicolumn{4}{|p{16cm}|}{%
\begin{enumerate}\item Introdução ao aprendizado não supervisionado: definição, desafios e aplicações.
\item Análise exploratória de dados: visualização, estatísticas descritivas e pré-processamento.
\item Técnicas de agrupamento: k-means, k-medoids, DBSCAN, aglomerativo hierárquico.
\item Métricas para avaliação de agrupamentos: silhueta, SSE, Davies-Bouldin.
\item Modelos baseados em mistura: Gaussian Mixture Models (GMM).
\item Redução de dimensionalidade: PCA (Análise de Componentes Principais), t-SNE, UMAP.
\item Extração de características e embeddings para dados complexos.
\item Detecção de anomalias: outliers, isolamento de florestas, técnicas baseadas em distância.
\item Tópicos avançados: agrupamento em dados de alta dimensão e em grandes volumes.
\item Implementações práticas com Python (scikit-learn, seaborn, matplotlib, numpy, pandas).\end{enumerate}}\\
\multicolumn{4}{|p{16cm}|}{}\\
\multicolumn{4}{|p{16cm}|}{}\\
\multicolumn{4}{|p{16cm}|}{\vspace{-1cm}}\\
\multicolumn{4}{|p{16cm}|}{}\\
\hline
\multicolumn{4}{|p{16cm}|}{Bibliografia Básica:}\\
\multicolumn{4}{|p{16cm}|}{%
\begin{itemize}\item GERON, Aurélien. Mãos à obra: aprendizado de máquina com Scikit-Learn, Keras e TensorFlow. 2. ed. São Paulo: Novatec, 2020. E-book. Disponível em: https://plataforma.bvirtual.com.br. Acesso em: jun/2025.
\item MURPHY, Kevin P. Probabilistic Machine Learning: An Introduction. Cambridge: MIT Press, 2022.
\item JAMES, Gareth et al. An Introduction to Statistical Learning. 2. ed. New York: Springer, 2021. (Capítulos sobre clustering e PCA).\end{itemize}}\\
\multicolumn{4}{|p{16cm}|}{}\\
\hline
\multicolumn{4}{|p{16cm}|}{Bibliografia Complementar:}\\
\multicolumn{4}{|p{16cm}|}{%
\begin{itemize}\item HASTIE, Trevor; TIBSHIRANI, Robert; FRIEDMAN, Jerome. The Elements of Statistical Learning. 2. ed. New York: Springer, 2009.
\item MÜLLER, Andreas C.; GUIDO, Sarah. Introduction to Machine Learning with Python. 1. ed. O'Reilly, 2016.
\item BISHOP, Christopher M. Pattern Recognition and Machine Learning. New York: Springer, 2006.
\item AGARWAL, Charu C.; ZHAO, Zhi-Hua. Advanced Methods for Unsupervised Learning. Springer, 2023.
\item OLIVEIRA, Hélio C. V. de. Mineração de dados: conceitos, algoritmos, aplicações. 2. ed. Rio de Janeiro: LTC, 2018. E-book. Disponível em: https://plataforma.bvirtual.com.br. Acesso em: jun/2025.\end{itemize}}\\
\hline
\end{longtable}
\end{center}

\clearpage
\begin{center}
\begin{tabular}{lccr}
 \multirow{3}{*}{\includegraphics[height=2.7cm]{brasao.png}} &
 \multicolumn{2}{c}{\bfseries UNIVERSIDADE FEDERAL DE OURO PRETO} &
 \ \ \ \ \multirow{3}{*}{\includegraphics[height=2.7cm]{ufop.png}} \\
 & \multicolumn{2}{c}{\bfseries PRÓ-REITORIA DE GRADUAÇÃO} & \\
 & \multicolumn{2}{c}{} & \\
 & \multicolumn{2}{c}{\Large\bfseries PROGRAMA DE DISCIPLINA} & \\
\end{tabular}
\end{center}

\begin{center}
\begin{longtable}{|p{4cm}|p{4cm}|p{4cm}|p{4cm}|}
\hline
\multicolumn{3}{|p{12cm}|}{Nome do Componente Curricular em Português:} &
\multicolumn{1}{p{4cm}|}{Código:} \\ 
\multicolumn{3}{|p{12cm}|}{\textbf{Metodologia Científica para Inteligência Artificial}} &
\textbf{BIA011}\\ 
\multicolumn{3}{|p{12cm}|}{Nome do Componente Curricular em Inglês:} & \\ 
\multicolumn{3}{|p{12cm}|}{\textbf{Scientific Methodology for Artificial Intelligence}} & \\ 
\hline
\multicolumn{3}{|p{12cm}|}{Nome e Sigla do Departamento} & Unidade Acadêmica: \\ 
\multicolumn{3}{|p{12cm}|}{Departamento de Computação (DECOM)} & {ICEB} \\ 
\hline
\multicolumn{4}{|p{16cm}|}{Modalidade de Oferta:
[X] presencial \hspace{1cm}
[ ] à distância}\\
\hline
\multicolumn{2}{|p{8cm}|}{Carga horária semestral} &
\multicolumn{2}{p{8cm}|}{Carga horária semanal}\\
\hline
\multicolumn{1}{|p{4cm}|}{Total} &
\multicolumn{1}{p{4cm}|}{Extensionista} &
\multicolumn{1}{p{4cm}|}{Teórica} &
\multicolumn{1}{p{4cm}|}{Prática} \\ 
\multicolumn{1}{|p{4cm}|}{60\,horas} &
\multicolumn{1}{p{4cm}|}{0\;horas} &
\multicolumn{1}{p{4cm}|}{4\;horas/aula} &
\multicolumn{1}{p{4cm}|}{0\;horas/aula} \\ 
\hline
\multicolumn{4}{|p{16cm}|}{Ementa:}\\
\multicolumn{4}{|p{16cm}|}{}\\
\multicolumn{4}{|p{16cm}|}{Conceitos e práticas da metodologia científica aplicadas à pesquisa em Inteligência Artificial. Estrutura de projetos científicos e tecnológicos. Ética na pesquisa em IA. Técnicas de leitura, escrita e comunicação científica. Avaliação crítica de artigos e reprodutibilidade científica.}\\
\multicolumn{4}{|p{16cm}|}{}\\
\hline
\multicolumn{4}{|p{16cm}|}{Conteúdo programático:}\\
\multicolumn{4}{|p{16cm}|}{%
\begin{enumerate}\item Fundamentos da ciência e do método científico: indução, dedução, hipótese e experimentação.
\item Tipos de pesquisa: básica, aplicada, exploratória, experimental e computacional.
\item Elaboração de problemas e hipóteses de pesquisa em IA e Computação.
\item Construção e avaliação de modelos e experimentos em IA.
\item Estrutura de projetos e relatórios científicos (TCCs, ICs, artigos).
\item Normas e padrões de escrita acadêmica (ABNT, IEEE, ACM).
\item Busca e avaliação de fontes bibliográficas: bases de dados, periódicos e conferências em IA.
\item Leitura crítica de artigos científicos: como interpretar e discutir resultados.
\item Reprodutibilidade e transparência na pesquisa em IA.
\item Ética em pesquisa científica: plágio, integridade, viés algorítmico e responsabilidade social.\end{enumerate}}\\
\multicolumn{4}{|p{16cm}|}{}\\
\multicolumn{4}{|p{16cm}|}{}\\
\multicolumn{4}{|p{16cm}|}{\vspace{-1cm}}\\
\multicolumn{4}{|p{16cm}|}{}\\
\hline
\multicolumn{4}{|p{16cm}|}{Bibliografia Básica:}\\
\multicolumn{4}{|p{16cm}|}{%
\begin{itemize}\item GIL, Antonio Carlos. Como elaborar projetos de pesquisa. 7. ed. São Paulo: Atlas, 2019. E-book. Disponível em: https://plataforma.bvirtual.com.br. Acesso em: jun/2025.
\item MARCONI, Marina de Andrade; LAKATOS, Eva Maria. Fundamentos de metodologia científica. 8. ed. São Paulo: Atlas, 2017. E-book. Disponível em: https://plataforma.bvirtual.com.br. Acesso em: jun/2025.
\item DIAS, Ana Francisca Pinto et al. (org.). Os direitos humanos e a ética na era da inteligência artificial. Indaiatuba, SP: Foco, 2023. E-book. Disponível em: https://plataforma.bvirtual.com.br. Acesso em: jun/2025.\end{itemize}}\\
\multicolumn{4}{|p{16cm}|}{}\\
\hline
\multicolumn{4}{|p{16cm}|}{Bibliografia Complementar:}\\
\multicolumn{4}{|p{16cm}|}{%
\begin{itemize}\item WAZLAWICK, Raul Sidnei. Metodologia de pesquisa para ciência da computação. 1. ed. Rio de Janeiro: Elsevier, 2009.
\item KAPLAN, David M. Philosophy of Technology. 2. ed. New York: Routledge, 2017.
\item CHALMERS, Alan F. O que é ciência afinal? 4. ed. São Paulo: Brasiliense, 2020.
\item ASSOCIAÇÃO BRASILEIRA DE NORMAS TÉCNICAS. NBR 6023, NBR 10520 e demais normas para trabalhos acadêmicos. Disponível em: https://www.abnt.org.br. Acesso em: jun/2025.
\item RUSSELL, Stuart; NORVIG, Peter. Inteligência artificial. 3. ed. São Paulo: Pearson, 2013. (Capítulo sobre perspectivas filosóficas e éticas em IA).\end{itemize}}\\
\hline
\end{longtable}
\end{center}

\clearpage
\begin{center}
\begin{tabular}{lccr}
 \multirow{3}{*}{\includegraphics[height=2.7cm]{brasao.png}} &
 \multicolumn{2}{c}{\bfseries UNIVERSIDADE FEDERAL DE OURO PRETO} &
 \ \ \ \ \multirow{3}{*}{\includegraphics[height=2.7cm]{ufop.png}} \\
 & \multicolumn{2}{c}{\bfseries PRÓ-REITORIA DE GRADUAÇÃO} & \\
 & \multicolumn{2}{c}{} & \\
 & \multicolumn{2}{c}{\Large\bfseries PROGRAMA DE DISCIPLINA} & \\
\end{tabular}
\end{center}

\begin{center}
\begin{longtable}{|p{4cm}|p{4cm}|p{4cm}|p{4cm}|}
\hline
\multicolumn{3}{|p{12cm}|}{Nome do Componente Curricular em Português:} &
\multicolumn{1}{p{4cm}|}{Código:} \\ 
\multicolumn{3}{|p{12cm}|}{\textbf{Modelos de Linguagem em Larga Escala}} &
\textbf{BIA012}\\ 
\multicolumn{3}{|p{12cm}|}{Nome do Componente Curricular em Inglês:} & \\ 
\multicolumn{3}{|p{12cm}|}{\textbf{Large-Scale Language Models}} & \\ 
\hline
\multicolumn{3}{|p{12cm}|}{Nome e Sigla do Departamento} & Unidade Acadêmica: \\ 
\multicolumn{3}{|p{12cm}|}{Departamento de Computação (DECOM)} & {ICEB} \\ 
\hline
\multicolumn{4}{|p{16cm}|}{Modalidade de Oferta:
[X] presencial \hspace{1cm}
[ ] à distância}\\
\hline
\multicolumn{2}{|p{8cm}|}{Carga horária semestral} &
\multicolumn{2}{p{8cm}|}{Carga horária semanal}\\
\hline
\multicolumn{1}{|p{4cm}|}{Total} &
\multicolumn{1}{p{4cm}|}{Extensionista} &
\multicolumn{1}{p{4cm}|}{Teórica} &
\multicolumn{1}{p{4cm}|}{Prática} \\ 
\multicolumn{1}{|p{4cm}|}{60\,horas} &
\multicolumn{1}{p{4cm}|}{0\;horas} &
\multicolumn{1}{p{4cm}|}{4\;horas/aula} &
\multicolumn{1}{p{4cm}|}{0\;horas/aula} \\ 
\hline
\multicolumn{4}{|p{16cm}|}{Ementa:}\\
\multicolumn{4}{|p{16cm}|}{}\\
\multicolumn{4}{|p{16cm}|}{Estudo de técnicas avançadas em Processamento de Linguagem Natural (PLN) e modelos de linguagem de larga escala (LLMs). Arquiteturas baseadas em transformadores, pré-treinamento, fine-tuning, geração de texto, aplicações e desafios éticos.}\\
\multicolumn{4}{|p{16cm}|}{}\\
\hline
\multicolumn{4}{|p{16cm}|}{Conteúdo programático:}\\
\multicolumn{4}{|p{16cm}|}{%
\begin{enumerate}\item Introdução ao Processamento de Linguagem Natural: história e desafios.
\item Representação de texto: embeddings tradicionais (TF-IDF, word2vec, GloVe) e contextualizados (ELMo, BERT).
\item Modelos de linguagem clássicos: n-gramas, Markov e RNNs.
\item Arquitetura Transformer: atenção, codificadores e decodificadores.
\item Modelos de linguagem de larga escala: GPT, BERT, T5 e variantes.
\item Pré-treinamento e fine-tuning: técnicas, datasets e estratégias.
\item Avaliação de modelos de linguagem: perplexidade, BLEU, ROUGE e outras métricas.
\item Aplicações práticas: chatbots, tradução automática, sumarização e análise de sentimento.
\item Desafios e limitações: vieses, ética, consumo computacional e explicabilidade.
\item Tendências atuais e pesquisas emergentes em PLN e LLMs.\end{enumerate}}\\
\multicolumn{4}{|p{16cm}|}{}\\
\multicolumn{4}{|p{16cm}|}{}\\
\multicolumn{4}{|p{16cm}|}{\vspace{-1cm}}\\
\multicolumn{4}{|p{16cm}|}{}\\
\hline
\multicolumn{4}{|p{16cm}|}{Bibliografia Básica:}\\
\multicolumn{4}{|p{16cm}|}{%
\begin{itemize}\item JURAFSKY, Daniel; MARTIN, James H. Speech and Language Processing. 3. ed. Draft. 2023. Disponível em: https://web.stanford.edu/~jurafsky/slp3/. Acesso em: jun/2025.
\item GOODFELLOW, Ian; BENGIO, Yoshua; COURVILLE, Aaron. Deep Learning. Cambridge: MIT Press, 2016. (Capítulos sobre redes neurais e atenção).
\item VASWANI, Ashish et al. Attention is All You Need. In: Advances in Neural Information Processing Systems (NeurIPS), 2017.\end{itemize}}\\
\multicolumn{4}{|p{16cm}|}{}\\
\hline
\multicolumn{4}{|p{16cm}|}{Bibliografia Complementar:}\\
\multicolumn{4}{|p{16cm}|}{%
\begin{itemize}\item RADFORD, Alec et al. Language Models are Few-Shot Learners. OpenAI, 2020. Disponível em: https://arxiv.org/abs/2005.14165.
\item DEVLIN, Jacob et al. BERT: Pre-training of Deep Bidirectional Transformers for Language Understanding. 2019. Disponível em: https://arxiv.org/abs/1810.04805.
\item CHOWDHURY, Gaurav. Natural Language Processing. 1. ed. Springer, 2021.
\item MANNING, Christopher D.; SCHÜTZE, Hinrich. Foundations of Statistical Natural Language Processing. Cambridge: MIT Press, 1999.
\item BROWN, Tom et al. GPT-3: Language Models are Few-Shot Learners. OpenAI, 2020.\end{itemize}}\\
\hline
\end{longtable}
\end{center}

\clearpage
\begin{center}
\begin{tabular}{lccr}
 \multirow{3}{*}{\includegraphics[height=2.7cm]{brasao.png}} &
 \multicolumn{2}{c}{\bfseries UNIVERSIDADE FEDERAL DE OURO PRETO} &
 \ \ \ \ \multirow{3}{*}{\includegraphics[height=2.7cm]{ufop.png}} \\
 & \multicolumn{2}{c}{\bfseries PRÓ-REITORIA DE GRADUAÇÃO} & \\
 & \multicolumn{2}{c}{} & \\
 & \multicolumn{2}{c}{\Large\bfseries PROGRAMA DE DISCIPLINA} & \\
\end{tabular}
\end{center}

\begin{center}
\begin{longtable}{|p{4cm}|p{4cm}|p{4cm}|p{4cm}|}
\hline
\multicolumn{3}{|p{12cm}|}{Nome do Componente Curricular em Português:} &
\multicolumn{1}{p{4cm}|}{Código:} \\ 
\multicolumn{3}{|p{12cm}|}{\textbf{UI e UX para Inteligência Artificial}} &
\textbf{BIA013}\\ 
\multicolumn{3}{|p{12cm}|}{Nome do Componente Curricular em Inglês:} & \\ 
\multicolumn{3}{|p{12cm}|}{\textbf{UI and UX for Artificial Intelligence}} & \\ 
\hline
\multicolumn{3}{|p{12cm}|}{Nome e Sigla do Departamento} & Unidade Acadêmica: \\ 
\multicolumn{3}{|p{12cm}|}{Departamento de Computação (DECOM)} & {ICEB} \\ 
\hline
\multicolumn{4}{|p{16cm}|}{Modalidade de Oferta:
[X] presencial \hspace{1cm}
[ ] à distância}\\
\hline
\multicolumn{2}{|p{8cm}|}{Carga horária semestral} &
\multicolumn{2}{p{8cm}|}{Carga horária semanal}\\
\hline
\multicolumn{1}{|p{4cm}|}{Total} &
\multicolumn{1}{p{4cm}|}{Extensionista} &
\multicolumn{1}{p{4cm}|}{Teórica} &
\multicolumn{1}{p{4cm}|}{Prática} \\ 
\multicolumn{1}{|p{4cm}|}{60\,horas} &
\multicolumn{1}{p{4cm}|}{0\;horas} &
\multicolumn{1}{p{4cm}|}{4\;horas/aula} &
\multicolumn{1}{p{4cm}|}{0\;horas/aula} \\ 
\hline
\multicolumn{4}{|p{16cm}|}{Ementa:}\\
\multicolumn{4}{|p{16cm}|}{}\\
\multicolumn{4}{|p{16cm}|}{Estudo dos princípios de design de interfaces de usuário (UI) e experiência do usuário (UX) aplicados a sistemas com Inteligência Artificial. Foco em usabilidade, acessibilidade, interação homem-máquina e avaliação de interfaces inteligentes.}\\
\multicolumn{4}{|p{16cm}|}{}\\
\hline
\multicolumn{4}{|p{16cm}|}{Conteúdo programático:}\\
\multicolumn{4}{|p{16cm}|}{%
\begin{enumerate}\item Fundamentos de UI e UX: conceitos, história e importância.
\item Princípios de design centrado no usuário (user-centered design).
\item Especificidades de UX para sistemas baseados em IA: transparência, confiança e controle do usuário.
\item Técnicas de prototipagem e wireframing para interfaces inteligentes.
\item Interação humano-IA: design de conversação, chatbots e assistentes virtuais.
\item Avaliação de usabilidade: métodos qualitativos e quantitativos.
\item Acessibilidade digital e inclusão em sistemas inteligentes.
\item Desafios éticos em UI/UX para IA: viés, privacidade e explicabilidade.
\item Ferramentas e frameworks para desenvolvimento de UI/UX em IA.
\item Estudos de caso: aplicações reais em IA e interfaces inteligentes.\end{enumerate}}\\
\multicolumn{4}{|p{16cm}|}{}\\
\multicolumn{4}{|p{16cm}|}{}\\
\multicolumn{4}{|p{16cm}|}{\vspace{-1cm}}\\
\multicolumn{4}{|p{16cm}|}{}\\
\hline
\multicolumn{4}{|p{16cm}|}{Bibliografia Básica:}\\
\multicolumn{4}{|p{16cm}|}{%
\begin{itemize}\item NORMAN, Donald A. The Design of Everyday Things. 2. ed. Basic Books, 2013.
\item COOPER, Alan; REIMANN, Robert; CRONIN, David. About Face: The Essentials of Interaction Design. 4. ed. Wiley, 2014.
\item SHNEIDERMAN, Ben; PIATKO, Catherine; COHEN, Jenifer. Designing the User Interface: Strategies for Effective Human-Computer Interaction. 6. ed. Pearson, 2016.\end{itemize}}\\
\multicolumn{4}{|p{16cm}|}{}\\
\hline
\multicolumn{4}{|p{16cm}|}{Bibliografia Complementar:}\\
\multicolumn{4}{|p{16cm}|}{%
\begin{itemize}\item MORRIS, Jenny. Designing for AI: Creating User Interfaces for Artificial Intelligence Systems. O'Reilly Media, 2021.
\item FOGEL, Karen; ESER, Daniel. UX Design for Artificial Intelligence Systems. 1. ed. Springer, 2020.
\item KLEIN, Julie; HACKER, Sue. Designing Bots: Creating Conversational Experiences. O'Reilly Media, 2017.
\item GREEN, Ben. The Ethical Challenges of AI in UX Design. MIT Press, 2022.
\item SAWYER, Steve. UX Strategy: How to Devise Innovative Digital Products that People Want. 2. ed. O'Reilly Media, 2015.\end{itemize}}\\
\hline
\end{longtable}
\end{center}

\clearpage
\begin{center}
\begin{tabular}{lccr}
 \multirow{3}{*}{\includegraphics[height=2.7cm]{brasao.png}} &
 \multicolumn{2}{c}{\bfseries UNIVERSIDADE FEDERAL DE OURO PRETO} &
 \ \ \ \ \multirow{3}{*}{\includegraphics[height=2.7cm]{ufop.png}} \\
 & \multicolumn{2}{c}{\bfseries PRÓ-REITORIA DE GRADUAÇÃO} & \\
 & \multicolumn{2}{c}{} & \\
 & \multicolumn{2}{c}{\Large\bfseries PROGRAMA DE DISCIPLINA} & \\
\end{tabular}
\end{center}

\begin{center}
\begin{longtable}{|p{4cm}|p{4cm}|p{4cm}|p{4cm}|}
\hline
\multicolumn{3}{|p{12cm}|}{Nome do Componente Curricular em Português:} &
\multicolumn{1}{p{4cm}|}{Código:} \\ 
\multicolumn{3}{|p{12cm}|}{\textbf{Visão Computacional e Processamento de Imagem}} &
\textbf{BIA014}\\ 
\multicolumn{3}{|p{12cm}|}{Nome do Componente Curricular em Inglês:} & \\ 
\multicolumn{3}{|p{12cm}|}{\textbf{Computer Vision and Image Processing}} & \\ 
\hline
\multicolumn{3}{|p{12cm}|}{Nome e Sigla do Departamento} & Unidade Acadêmica: \\ 
\multicolumn{3}{|p{12cm}|}{Departamento de Computação (DECOM)} & {ICEB} \\ 
\hline
\multicolumn{4}{|p{16cm}|}{Modalidade de Oferta:
[X] presencial \hspace{1cm}
[ ] à distância}\\
\hline
\multicolumn{2}{|p{8cm}|}{Carga horária semestral} &
\multicolumn{2}{p{8cm}|}{Carga horária semanal}\\
\hline
\multicolumn{1}{|p{4cm}|}{Total} &
\multicolumn{1}{p{4cm}|}{Extensionista} &
\multicolumn{1}{p{4cm}|}{Teórica} &
\multicolumn{1}{p{4cm}|}{Prática} \\ 
\multicolumn{1}{|p{4cm}|}{60\,horas} &
\multicolumn{1}{p{4cm}|}{0\;horas} &
\multicolumn{1}{p{4cm}|}{4\;horas/aula} &
\multicolumn{1}{p{4cm}|}{0\;horas/aula} \\ 
\hline
\multicolumn{4}{|p{16cm}|}{Ementa:}\\
\multicolumn{4}{|p{16cm}|}{}\\
\multicolumn{4}{|p{16cm}|}{Introdução aos fundamentos da visão computacional e processamento de imagem digital. Técnicas para análise, transformação e interpretação de imagens. Aplicações em reconhecimento, segmentação e análise visual.}\\
\multicolumn{4}{|p{16cm}|}{}\\
\hline
\multicolumn{4}{|p{16cm}|}{Conteúdo programático:}\\
\multicolumn{4}{|p{16cm}|}{%
\begin{enumerate}\item Fundamentos da formação de imagens e percepção visual.
\item Representação digital de imagens: pixels, cores e formatos.
\item Operações básicas de processamento de imagens: filtragem, realce e transformação.
\item Detecção de bordas, contornos e extração de características.
\item Segmentação de imagens: métodos baseados em limiarização, region growing e clustering.
\item Morfologia matemática e processamento de formas.
\item Reconhecimento de padrões e classificação de imagens.
\item Visão computacional baseada em aprendizado: redes neurais convolucionais (CNNs).
\item Detecção e rastreamento de objetos em vídeo.
\item Aplicações práticas: reconhecimento facial, visão para robótica, análise médica.\end{enumerate}}\\
\multicolumn{4}{|p{16cm}|}{}\\
\multicolumn{4}{|p{16cm}|}{}\\
\multicolumn{4}{|p{16cm}|}{\vspace{-1cm}}\\
\multicolumn{4}{|p{16cm}|}{}\\
\hline
\multicolumn{4}{|p{16cm}|}{Bibliografia Básica:}\\
\multicolumn{4}{|p{16cm}|}{%
\begin{itemize}\item GONZALEZ, Rafael C.; WOODS, Richard E. Processamento de Imagens Digitais. 4. ed. Pearson, 2018.
\item SZE, Vivienne et al. Efficient Processing of Deep Neural Networks: A Tutorial and Survey. Proceedings of the IEEE, 2017.
\item SZAJDAK, Marek. Introdução à Visão Computacional. 1. ed. LTC, 2021.\end{itemize}}\\
\multicolumn{4}{|p{16cm}|}{}\\
\hline
\multicolumn{4}{|p{16cm}|}{Bibliografia Complementar:}\\
\multicolumn{4}{|p{16cm}|}{%
\begin{itemize}\item GOODFELLOW, Ian; BENGIO, Yoshua; COURVILLE, Aaron. Deep Learning. Cambridge: MIT Press, 2016. (Capítulo sobre CNNs).
\item RUSSELL, Stuart; NORVIG, Peter. Inteligência Artificial. 3. ed. Pearson, 2013. (Capítulo sobre visão computacional).
\item FORSYTHT, David A. Computer Vision: A Modern Approach. 2. ed. Pearson, 2011.
\item BALLARD, Dana H.; BROWN, Christopher M. Computer Vision. Prentice Hall, 1982.
\item ZHAO, Sheng et al. Deep Learning for Computer Vision: A Brief Review. IEEE Access, 2019.\end{itemize}}\\
\hline
\end{longtable}
\end{center}

\clearpage
\begin{center}
\begin{tabular}{lccr}
 \multirow{3}{*}{\includegraphics[height=2.7cm]{brasao.png}} &
 \multicolumn{2}{c}{\bfseries UNIVERSIDADE FEDERAL DE OURO PRETO} &
 \ \ \ \ \multirow{3}{*}{\includegraphics[height=2.7cm]{ufop.png}} \\
 & \multicolumn{2}{c}{\bfseries PRÓ-REITORIA DE GRADUAÇÃO} & \\
 & \multicolumn{2}{c}{} & \\
 & \multicolumn{2}{c}{\Large\bfseries PROGRAMA DE DISCIPLINA} & \\
\end{tabular}
\end{center}

\begin{center}
\begin{longtable}{|p{4cm}|p{4cm}|p{4cm}|p{4cm}|}
\hline
\multicolumn{3}{|p{12cm}|}{Nome do Componente Curricular em Português:} &
\multicolumn{1}{p{4cm}|}{Código:} \\ 
\multicolumn{3}{|p{12cm}|}{\textbf{Computação de Alto Desempenho}} &
\textbf{BIA015}\\ 
\multicolumn{3}{|p{12cm}|}{Nome do Componente Curricular em Inglês:} & \\ 
\multicolumn{3}{|p{12cm}|}{\textbf{High Performance Computing}} & \\ 
\hline
\multicolumn{3}{|p{12cm}|}{Nome e Sigla do Departamento} & Unidade Acadêmica: \\ 
\multicolumn{3}{|p{12cm}|}{Departamento de Computação (DECOM)} & {ICEB} \\ 
\hline
\multicolumn{4}{|p{16cm}|}{Modalidade de Oferta:
[X] presencial \hspace{1cm}
[ ] à distância}\\
\hline
\multicolumn{2}{|p{8cm}|}{Carga horária semestral} &
\multicolumn{2}{p{8cm}|}{Carga horária semanal}\\
\hline
\multicolumn{1}{|p{4cm}|}{Total} &
\multicolumn{1}{p{4cm}|}{Extensionista} &
\multicolumn{1}{p{4cm}|}{Teórica} &
\multicolumn{1}{p{4cm}|}{Prática} \\ 
\multicolumn{1}{|p{4cm}|}{60\,horas} &
\multicolumn{1}{p{4cm}|}{0\;horas} &
\multicolumn{1}{p{4cm}|}{4\;horas/aula} &
\multicolumn{1}{p{4cm}|}{0\;horas/aula} \\ 
\hline
\multicolumn{4}{|p{16cm}|}{Ementa:}\\
\multicolumn{4}{|p{16cm}|}{}\\
\multicolumn{4}{|p{16cm}|}{Estudo das arquiteturas, técnicas e ferramentas para computação de alto desempenho (HPC). Paralelismo, programação concorrente, arquiteturas multicore e distribuídas, otimização de código e aplicações científicas e de engenharia.}\\
\multicolumn{4}{|p{16cm}|}{}\\
\hline
\multicolumn{4}{|p{16cm}|}{Conteúdo programático:}\\
\multicolumn{4}{|p{16cm}|}{%
\begin{enumerate}\item Introdução à computação de alto desempenho: conceitos e aplicações.
\item Arquiteturas paralelas: SIMD, MIMD, multiprocessadores e clusters.
\item Memória compartilhada vs memória distribuída.
\item Programação paralela: paradigmas e modelos (threads, MPI, OpenMP).
\item Sincronização e comunicação entre processos.
\item Técnicas de otimização e escalabilidade de programas.
\item Computação em GPU: conceitos e programação com CUDA/OpenCL.
\item Ferramentas e bibliotecas para HPC.
\item Gerenciamento de recursos e ambientes de execução HPC.
\item Estudos de caso: simulações científicas, aprendizado de máquina em HPC, processamento de grandes volumes de dados.\end{enumerate}}\\
\multicolumn{4}{|p{16cm}|}{}\\
\multicolumn{4}{|p{16cm}|}{}\\
\multicolumn{4}{|p{16cm}|}{\vspace{-1cm}}\\
\multicolumn{4}{|p{16cm}|}{}\\
\hline
\multicolumn{4}{|p{16cm}|}{Bibliografia Básica:}\\
\multicolumn{4}{|p{16cm}|}{%
\begin{itemize}\item PACHECO, Peter S. An Introduction to Parallel Programming. 2. ed. Morgan Kaufmann, 2011.
\item GRAMA, Ananth; GUPTA, Anshul; KARYPIS, George; KUMAR, Vipin. Introduction to Parallel Computing. 2. ed. Pearson, 2003.
\item KIRKPATRICK, Scott. CUDA Programming: A Developer's Guide to Parallel Computing with GPUs. Addison-Wesley, 2013.\end{itemize}}\\
\multicolumn{4}{|p{16cm}|}{}\\
\hline
\multicolumn{4}{|p{16cm}|}{Bibliografia Complementar:}\\
\multicolumn{4}{|p{16cm}|}{%
\begin{itemize}\item HERLIHY, Maurice; SHAVIT, Nir. The Art of Multiprocessor Programming. Revised Reprint. Morgan Kaufmann, 2011.
\item CHANDRA, Ramesh; MENON, Ravi; FINKELSTEIN, Albert. Parallel Programming in OpenMP. Morgan Kaufmann, 2001.
\item RAJ, Rakesh. High Performance Computing: Paradigm and Infrastructure. Wiley, 2017.
\item KARYPIS, George; GUPTA, Anshul. Parallel Programming for Modern HPC Architectures. Springer, 2020.
\item RUSSELL, Stuart; NORVIG, Peter. Inteligência Artificial. 3. ed. Pearson, 2013. (Capítulo sobre paralelismo em IA).\end{itemize}}\\
\hline
\end{longtable}
\end{center}

\clearpage
\begin{center}
\begin{tabular}{lccr}
 \multirow{3}{*}{\includegraphics[height=2.7cm]{brasao.png}} &
 \multicolumn{2}{c}{\bfseries UNIVERSIDADE FEDERAL DE OURO PRETO} &
 \ \ \ \ \multirow{3}{*}{\includegraphics[height=2.7cm]{ufop.png}} \\
 & \multicolumn{2}{c}{\bfseries PRÓ-REITORIA DE GRADUAÇÃO} & \\
 & \multicolumn{2}{c}{} & \\
 & \multicolumn{2}{c}{\Large\bfseries PROGRAMA DE DISCIPLINA} & \\
\end{tabular}
\end{center}

\begin{center}
\begin{longtable}{|p{4cm}|p{4cm}|p{4cm}|p{4cm}|}
\hline
\multicolumn{3}{|p{12cm}|}{Nome do Componente Curricular em Português:} &
\multicolumn{1}{p{4cm}|}{Código:} \\ 
\multicolumn{3}{|p{12cm}|}{\textbf{Processamento de Dados Massivos}} &
\textbf{BIA016}\\ 
\multicolumn{3}{|p{12cm}|}{Nome do Componente Curricular em Inglês:} & \\ 
\multicolumn{3}{|p{12cm}|}{\textbf{Massive Data Processing}} & \\ 
\hline
\multicolumn{3}{|p{12cm}|}{Nome e Sigla do Departamento} & Unidade Acadêmica: \\ 
\multicolumn{3}{|p{12cm}|}{Departamento de Computação (DECOM)} & {ICEB} \\ 
\hline
\multicolumn{4}{|p{16cm}|}{Modalidade de Oferta:
[X] presencial \hspace{1cm}
[ ] à distância}\\
\hline
\multicolumn{2}{|p{8cm}|}{Carga horária semestral} &
\multicolumn{2}{p{8cm}|}{Carga horária semanal}\\
\hline
\multicolumn{1}{|p{4cm}|}{Total} &
\multicolumn{1}{p{4cm}|}{Extensionista} &
\multicolumn{1}{p{4cm}|}{Teórica} &
\multicolumn{1}{p{4cm}|}{Prática} \\ 
\multicolumn{1}{|p{4cm}|}{60\,horas} &
\multicolumn{1}{p{4cm}|}{0\;horas} &
\multicolumn{1}{p{4cm}|}{4\;horas/aula} &
\multicolumn{1}{p{4cm}|}{0\;horas/aula} \\ 
\hline
\multicolumn{4}{|p{16cm}|}{Ementa:}\\
\multicolumn{4}{|p{16cm}|}{}\\
\multicolumn{4}{|p{16cm}|}{Estudo das tecnologias, arquiteturas e técnicas para processamento eficiente de grandes volumes de dados. Sistemas distribuídos, frameworks para Big Data, processamento em lote e em tempo real, armazenamento e análise de dados massivos.}\\
\multicolumn{4}{|p{16cm}|}{}\\
\hline
\multicolumn{4}{|p{16cm}|}{Conteúdo programático:}\\
\multicolumn{4}{|p{16cm}|}{%
\begin{enumerate}\item Conceitos e desafios do Big Data: volume, variedade, velocidade e veracidade.
\item Arquiteturas para processamento de dados massivos: sistemas distribuídos e em nuvem.
\item Modelos de programação paralela: MapReduce e suas variações.
\item Frameworks para Big Data: Hadoop, Spark, Flink e outros.
\item Armazenamento escalável: sistemas de arquivos distribuídos e bancos NoSQL.
\item Processamento em tempo real: stream processing, Apache Kafka e ferramentas associadas.
\item Técnicas de pré-processamento, limpeza e transformação de dados massivos.
\item Análise e mineração de dados em larga escala.
\item Gerenciamento de clusters e orquestração de tarefas.
\item Casos de uso: aplicações em indústria, ciência e serviços.\end{enumerate}}\\
\multicolumn{4}{|p{16cm}|}{}\\
\multicolumn{4}{|p{16cm}|}{}\\
\multicolumn{4}{|p{16cm}|}{\vspace{-1cm}}\\
\multicolumn{4}{|p{16cm}|}{}\\
\hline
\multicolumn{4}{|p{16cm}|}{Bibliografia Básica:}\\
\multicolumn{4}{|p{16cm}|}{%
\begin{itemize}\item MARRS, Anthony et al. Big Data: Principles and best practices of scalable real-time data systems. Manning Publications, 2015.
\item WHITE, Tom. Hadoop: The Definitive Guide. 4. ed. O'Reilly Media, 2015.
\item ZAHARIA, Matei et al. Learning Spark: Lightning-Fast Big Data Analysis. 2. ed. O'Reilly Media, 2016.\end{itemize}}\\
\multicolumn{4}{|p{16cm}|}{}\\
\hline
\multicolumn{4}{|p{16cm}|}{Bibliografia Complementar:}\\
\multicolumn{4}{|p{16cm}|}{%
\begin{itemize}\item DEAN, Jeffrey; GHEMAWAT, Sanjay. MapReduce: Simplified Data Processing on Large Clusters. Communications of the ACM, 2008.
\item KLEPPMANN, Martin. Designing Data-Intensive Applications. O'Reilly Media, 2017.
\item KARAU, Holden et al. High Performance Spark. O'Reilly Media, 2017.
\item AGRAWAL, Rakesh; IMIELIŃSKI, Tomasz; SWAMI, Arun. Mining Association Rules between Sets of Items in Large Databases. ACM SIGMOD Record, 1993.
\item WHITE, Tom. Hadoop: The Definitive Guide. O'Reilly Media, 2012.\end{itemize}}\\
\hline
\end{longtable}
\end{center}

\clearpage
\begin{center}
\begin{tabular}{lccr}
 \multirow{3}{*}{\includegraphics[height=2.7cm]{brasao.png}} &
 \multicolumn{2}{c}{\bfseries UNIVERSIDADE FEDERAL DE OURO PRETO} &
 \ \ \ \ \multirow{3}{*}{\includegraphics[height=2.7cm]{ufop.png}} \\
 & \multicolumn{2}{c}{\bfseries PRÓ-REITORIA DE GRADUAÇÃO} & \\
 & \multicolumn{2}{c}{} & \\
 & \multicolumn{2}{c}{\Large\bfseries PROGRAMA DE DISCIPLINA} & \\
\end{tabular}
\end{center}

\begin{center}
\begin{longtable}{|p{4cm}|p{4cm}|p{4cm}|p{4cm}|}
\hline
\multicolumn{3}{|p{12cm}|}{Nome do Componente Curricular em Português:} &
\multicolumn{1}{p{4cm}|}{Código:} \\ 
\multicolumn{3}{|p{12cm}|}{\textbf{Processamento de Áudio}} &
\textbf{BIA017}\\ 
\multicolumn{3}{|p{12cm}|}{Nome do Componente Curricular em Inglês:} & \\ 
\multicolumn{3}{|p{12cm}|}{\textbf{Audio Processing}} & \\ 
\hline
\multicolumn{3}{|p{12cm}|}{Nome e Sigla do Departamento} & Unidade Acadêmica: \\ 
\multicolumn{3}{|p{12cm}|}{Departamento de Computação (DECOM)} & {ICEB} \\ 
\hline
\multicolumn{4}{|p{16cm}|}{Modalidade de Oferta:
[X] presencial \hspace{1cm}
[ ] à distância}\\
\hline
\multicolumn{2}{|p{8cm}|}{Carga horária semestral} &
\multicolumn{2}{p{8cm}|}{Carga horária semanal}\\
\hline
\multicolumn{1}{|p{4cm}|}{Total} &
\multicolumn{1}{p{4cm}|}{Extensionista} &
\multicolumn{1}{p{4cm}|}{Teórica} &
\multicolumn{1}{p{4cm}|}{Prática} \\ 
\multicolumn{1}{|p{4cm}|}{60\,horas} &
\multicolumn{1}{p{4cm}|}{0\;horas} &
\multicolumn{1}{p{4cm}|}{4\;horas/aula} &
\multicolumn{1}{p{4cm}|}{0\;horas/aula} \\ 
\hline
\multicolumn{4}{|p{16cm}|}{Ementa:}\\
\multicolumn{4}{|p{16cm}|}{}\\
\multicolumn{4}{|p{16cm}|}{Introdução aos conceitos e técnicas fundamentais para o processamento digital de sinais de áudio. Análise, transformação, filtragem e síntese de áudio aplicados em sistemas de reconhecimento, síntese sonora e IA.}\\
\multicolumn{4}{|p{16cm}|}{}\\
\hline
\multicolumn{4}{|p{16cm}|}{Conteúdo programático:}\\
\multicolumn{4}{|p{16cm}|}{%
\begin{enumerate}\item Modelos acústicos tradicionais, misturas de gaussianas e cadeias ocultas de markov.
\item Representações de sinais para áudio e voz.
\item Arquiteturas de redes neurais para reconhecimento de fala.
\item Modelos de encoder, vocoder e arquiteturas para sintetização de voz.
\item Arquiteturas para aplicações em música\end{enumerate}}\\
\multicolumn{4}{|p{16cm}|}{}\\
\multicolumn{4}{|p{16cm}|}{}\\
\multicolumn{4}{|p{16cm}|}{\vspace{-1cm}}\\
\multicolumn{4}{|p{16cm}|}{}\\
\hline
\multicolumn{4}{|p{16cm}|}{Bibliografia Básica:}\\
\multicolumn{4}{|p{16cm}|}{%
\begin{itemize}\item KAMATH, U. L.; WHITAKER, J. Deep learning for NLP and speech recognition. Springer Nature, 2019.
\item DONG, Y.; DENG, Y. Automatic speech recognition. Springer London Limited, 2016.
\item GOPI, E. S. Digital speech processing using Matlab. New Delhi: Springer, 2014. xvi, 182 p. ISBN 9788132216766.\end{itemize}}\\
\multicolumn{4}{|p{16cm}|}{}\\
\hline
\multicolumn{4}{|p{16cm}|}{Bibliografia Complementar:}\\
\multicolumn{4}{|p{16cm}|}{%
\begin{itemize}\item CAMASTRA, F.; VINCIARELLI, A. Machine learning for audio, image and video analysis: theory and applications. Springer, 2015.
\item STEVENS, E.; Antiga, L. Deep Learning with Pytorch. Manning Publications, 2020.
\item GALEONE, P. Hands-On Neural Networks with TensorFlow 2.0: Understand TensorFlow, from static graph to eager execution, and design neural networks. Packt Publishing, 2019.\end{itemize}}\\
\hline
\end{longtable}
\end{center}

\clearpage
\begin{center}
\begin{tabular}{lccr}
 \multirow{3}{*}{\includegraphics[height=2.7cm]{brasao.png}} &
 \multicolumn{2}{c}{\bfseries UNIVERSIDADE FEDERAL DE OURO PRETO} &
 \ \ \ \ \multirow{3}{*}{\includegraphics[height=2.7cm]{ufop.png}} \\
 & \multicolumn{2}{c}{\bfseries PRÓ-REITORIA DE GRADUAÇÃO} & \\
 & \multicolumn{2}{c}{} & \\
 & \multicolumn{2}{c}{\Large\bfseries PROGRAMA DE DISCIPLINA} & \\
\end{tabular}
\end{center}

\begin{center}
\begin{longtable}{|p{4cm}|p{4cm}|p{4cm}|p{4cm}|}
\hline
\multicolumn{3}{|p{12cm}|}{Nome do Componente Curricular em Português:} &
\multicolumn{1}{p{4cm}|}{Código:} \\ 
\multicolumn{3}{|p{12cm}|}{\textbf{Internet das Coisas}} &
\textbf{BIA018}\\ 
\multicolumn{3}{|p{12cm}|}{Nome do Componente Curricular em Inglês:} & \\ 
\multicolumn{3}{|p{12cm}|}{\textbf{Internet of Things}} & \\ 
\hline
\multicolumn{3}{|p{12cm}|}{Nome e Sigla do Departamento} & Unidade Acadêmica: \\ 
\multicolumn{3}{|p{12cm}|}{Departamento de Computação (DECOM)} & {ICEB} \\ 
\hline
\multicolumn{4}{|p{16cm}|}{Modalidade de Oferta:
[X] presencial \hspace{1cm}
[ ] à distância}\\
\hline
\multicolumn{2}{|p{8cm}|}{Carga horária semestral} &
\multicolumn{2}{p{8cm}|}{Carga horária semanal}\\
\hline
\multicolumn{1}{|p{4cm}|}{Total} &
\multicolumn{1}{p{4cm}|}{Extensionista} &
\multicolumn{1}{p{4cm}|}{Teórica} &
\multicolumn{1}{p{4cm}|}{Prática} \\ 
\multicolumn{1}{|p{4cm}|}{60\,horas} &
\multicolumn{1}{p{4cm}|}{0\;horas} &
\multicolumn{1}{p{4cm}|}{4\;horas/aula} &
\multicolumn{1}{p{4cm}|}{0\;horas/aula} \\ 
\hline
\multicolumn{4}{|p{16cm}|}{Ementa:}\\
\multicolumn{4}{|p{16cm}|}{}\\
\multicolumn{4}{|p{16cm}|}{Introdução aos conceitos, arquiteturas e tecnologias da Internet das Coisas (IoT). Protocolos de comunicação, sensores, atuadores, segurança e aplicações inteligentes em ambientes conectados.}\\
\multicolumn{4}{|p{16cm}|}{}\\
\hline
\multicolumn{4}{|p{16cm}|}{Conteúdo programático:}\\
\multicolumn{4}{|p{16cm}|}{%
\begin{enumerate}\item Fundamentos da Internet das Coisas: definições, evolução e panorama atual.
\item Arquitetura IoT: camadas, dispositivos e sistemas embarcados.
\item Sensores e atuadores: tipos, funcionamento e integração.
\item Protocolos de comunicação IoT: MQTT, CoAP, HTTP, BLE, Zigbee.
\item Plataformas e ferramentas para desenvolvimento IoT.
\item Segurança e privacidade em ambientes IoT.
\item Coleta, transmissão e processamento de dados em IoT.
\item Integração com inteligência artificial e computação em nuvem.
\item Casos de uso: cidades inteligentes, agricultura de precisão, saúde e indústria 4.0.
\item Tendências e desafios futuros da Internet das Coisas.\end{enumerate}}\\
\multicolumn{4}{|p{16cm}|}{}\\
\multicolumn{4}{|p{16cm}|}{}\\
\multicolumn{4}{|p{16cm}|}{\vspace{-1cm}}\\
\multicolumn{4}{|p{16cm}|}{}\\
\hline
\multicolumn{4}{|p{16cm}|}{Bibliografia Básica:}\\
\multicolumn{4}{|p{16cm}|}{%
\begin{itemize}\item WILLETT, Richard; OLIVEIRA, Charalampos. Internet das Coisas: Conceitos, Tecnologias e Aplicações. Novatec, 2022.
\item VERGARA, Luiz Antônio. Internet das Coisas. 2. ed. Campus, 2020.
\item MINERVA, Roberto et al. Towards a definition of the Internet of Things (IoT). IEEE Internet Initiative, 2015.\end{itemize}}\\
\multicolumn{4}{|p{16cm}|}{}\\
\hline
\multicolumn{4}{|p{16cm}|}{Bibliografia Complementar:}\\
\multicolumn{4}{|p{16cm}|}{%
\begin{itemize}\item GUBBI, Jayavardhana et al. Internet of Things (IoT): A Vision, Architectural Elements, and Future Directions. Future Generation Computer Systems, 2013.
\item KURUNDRAN, S.; BHASKAR, P. IoT Protocols and Standards: An Overview. Springer, 2021.
\item RAY, Partha Pratim. Internet of Things: Architecture and Applications. Wiley, 2018.
\item SINGH, D.; ZHANG, H.; KUMAR, N. Internet of Things: Principles and Paradigms. Wiley, 2018.
\item STANKOVIC, John A. Research Directions for the Internet of Things. IEEE Internet of Things Journal, 2014.\end{itemize}}\\
\hline
\end{longtable}
\end{center}

\clearpage
\begin{center}
\begin{tabular}{lccr}
 \multirow{3}{*}{\includegraphics[height=2.7cm]{brasao.png}} &
 \multicolumn{2}{c}{\bfseries UNIVERSIDADE FEDERAL DE OURO PRETO} &
 \ \ \ \ \multirow{3}{*}{\includegraphics[height=2.7cm]{ufop.png}} \\
 & \multicolumn{2}{c}{\bfseries PRÓ-REITORIA DE GRADUAÇÃO} & \\
 & \multicolumn{2}{c}{} & \\
 & \multicolumn{2}{c}{\Large\bfseries PROGRAMA DE DISCIPLINA} & \\
\end{tabular}
\end{center}

\begin{center}
\begin{longtable}{|p{4cm}|p{4cm}|p{4cm}|p{4cm}|}
\hline
\multicolumn{3}{|p{12cm}|}{Nome do Componente Curricular em Português:} &
\multicolumn{1}{p{4cm}|}{Código:} \\ 
\multicolumn{3}{|p{12cm}|}{\textbf{Robótica}} &
\textbf{BIA019}\\ 
\multicolumn{3}{|p{12cm}|}{Nome do Componente Curricular em Inglês:} & \\ 
\multicolumn{3}{|p{12cm}|}{\textbf{Robotics}} & \\ 
\hline
\multicolumn{3}{|p{12cm}|}{Nome e Sigla do Departamento} & Unidade Acadêmica: \\ 
\multicolumn{3}{|p{12cm}|}{Departamento de Computação (DECOM)} & {ICEB} \\ 
\hline
\multicolumn{4}{|p{16cm}|}{Modalidade de Oferta:
[X] presencial \hspace{1cm}
[ ] à distância}\\
\hline
\multicolumn{2}{|p{8cm}|}{Carga horária semestral} &
\multicolumn{2}{p{8cm}|}{Carga horária semanal}\\
\hline
\multicolumn{1}{|p{4cm}|}{Total} &
\multicolumn{1}{p{4cm}|}{Extensionista} &
\multicolumn{1}{p{4cm}|}{Teórica} &
\multicolumn{1}{p{4cm}|}{Prática} \\ 
\multicolumn{1}{|p{4cm}|}{60\,horas} &
\multicolumn{1}{p{4cm}|}{0\;horas} &
\multicolumn{1}{p{4cm}|}{4\;horas/aula} &
\multicolumn{1}{p{4cm}|}{0\;horas/aula} \\ 
\hline
\multicolumn{4}{|p{16cm}|}{Ementa:}\\
\multicolumn{4}{|p{16cm}|}{}\\
\multicolumn{4}{|p{16cm}|}{Introdução aos conceitos fundamentais de robótica, incluindo cinemática, sensores, atuadores, controle, planejamento de movimento e inteligência artificial aplicada a sistemas robóticos autônomos.}\\
\multicolumn{4}{|p{16cm}|}{}\\
\hline
\multicolumn{4}{|p{16cm}|}{Conteúdo programático:}\\
\multicolumn{4}{|p{16cm}|}{%
\begin{enumerate}\item Histórico e aplicações da robótica.
\item Cinemática direta e inversa de manipuladores robóticos.
\item Modelagem dinâmica de robôs.
\item Sensores e atuadores em robótica.
\item Controle de movimento e sistemas embarcados.
\item Planejamento de trajetórias e navegação autônoma.
\item Percepção e fusão de dados sensoriais.
\item Robótica móvel e manipuladores colaborativos.
\item Introdução à inteligência artificial em robótica: aprendizado, planejamento e decisão.
\item Estudos de caso e aplicações práticas em indústria, saúde e automação.\end{enumerate}}\\
\multicolumn{4}{|p{16cm}|}{}\\
\multicolumn{4}{|p{16cm}|}{}\\
\multicolumn{4}{|p{16cm}|}{\vspace{-1cm}}\\
\multicolumn{4}{|p{16cm}|}{}\\
\hline
\multicolumn{4}{|p{16cm}|}{Bibliografia Básica:}\\
\multicolumn{4}{|p{16cm}|}{%
\begin{itemize}\item CRAIG, John J. Introduction to Robotics: Mechanics and Control. 4. ed. Pearson, 2013.
\item SICILIANO, Bruno; KORTEN, Oussama; KALTON, Lorenzo. Robotics: Modelling, Planning and Control. Springer, 2016.
\item KHALIL, Hassan K. Nonlinear Systems. 3. ed. Prentice Hall, 2002.\end{itemize}}\\
\multicolumn{4}{|p{16cm}|}{}\\
\hline
\multicolumn{4}{|p{16cm}|}{Bibliografia Complementar:}\\
\multicolumn{4}{|p{16cm}|}{%
\begin{itemize}\item SPIELBERGER, Uwe; SCHLEGEL, Christian. Robótica: fundamentos e aplicações. LTC, 2018.
\item THRUN, Sebastian; BURGARD, Wolfram; FOX, Dieter. Probabilistic Robotics. MIT Press, 2005.
\item MURRAY, Richard M.; LI, Zexiang; SASTRY, S. Shankar. A Mathematical Introduction to Robotic Manipulation. CRC Press, 1994.
\item FOX, Dieter; HENRY, Gregory. Introduction to Autonomous Robots. MIT Press, 2020.
\item SIEGEL, Scott. Artificial Intelligence for Robotics: Build Intelligent Robots that Perform Human Tasks. Packt, 2016.\end{itemize}}\\
\hline
\end{longtable}
\end{center}

\clearpage
\begin{center}
\begin{tabular}{lccr}
 \multirow{3}{*}{\includegraphics[height=2.7cm]{brasao.png}} &
 \multicolumn{2}{c}{\bfseries UNIVERSIDADE FEDERAL DE OURO PRETO} &
 \ \ \ \ \multirow{3}{*}{\includegraphics[height=2.7cm]{ufop.png}} \\
 & \multicolumn{2}{c}{\bfseries PRÓ-REITORIA DE GRADUAÇÃO} & \\
 & \multicolumn{2}{c}{} & \\
 & \multicolumn{2}{c}{\Large\bfseries PROGRAMA DE DISCIPLINA} & \\
\end{tabular}
\end{center}

\begin{center}
\begin{longtable}{|p{4cm}|p{4cm}|p{4cm}|p{4cm}|}
\hline
\multicolumn{3}{|p{12cm}|}{Nome do Componente Curricular em Português:} &
\multicolumn{1}{p{4cm}|}{Código:} \\ 
\multicolumn{3}{|p{12cm}|}{\textbf{Visualização e Interpretabilidade de Modelos de Inteligência Artificial}} &
\textbf{BIA020}\\ 
\multicolumn{3}{|p{12cm}|}{Nome do Componente Curricular em Inglês:} & \\ 
\multicolumn{3}{|p{12cm}|}{\textbf{Visualization and Interpretability of Artificial Intelligence Models}} & \\ 
\hline
\multicolumn{3}{|p{12cm}|}{Nome e Sigla do Departamento} & Unidade Acadêmica: \\ 
\multicolumn{3}{|p{12cm}|}{Departamento de Computação (DECOM)} & {ICEB} \\ 
\hline
\multicolumn{4}{|p{16cm}|}{Modalidade de Oferta:
[X] presencial \hspace{1cm}
[ ] à distância}\\
\hline
\multicolumn{2}{|p{8cm}|}{Carga horária semestral} &
\multicolumn{2}{p{8cm}|}{Carga horária semanal}\\
\hline
\multicolumn{1}{|p{4cm}|}{Total} &
\multicolumn{1}{p{4cm}|}{Extensionista} &
\multicolumn{1}{p{4cm}|}{Teórica} &
\multicolumn{1}{p{4cm}|}{Prática} \\ 
\multicolumn{1}{|p{4cm}|}{60\,horas} &
\multicolumn{1}{p{4cm}|}{0\;horas} &
\multicolumn{1}{p{4cm}|}{4\;horas/aula} &
\multicolumn{1}{p{4cm}|}{0\;horas/aula} \\ 
\hline
\multicolumn{4}{|p{16cm}|}{Ementa:}\\
\multicolumn{4}{|p{16cm}|}{}\\
\multicolumn{4}{|p{16cm}|}{Estudo de métodos e técnicas para visualização, interpretação e explicação de modelos de inteligência artificial, visando maior transparência, confiabilidade e compreensão dos resultados por humanos.}\\
\multicolumn{4}{|p{16cm}|}{}\\
\hline
\multicolumn{4}{|p{16cm}|}{Conteúdo programático:}\\
\multicolumn{4}{|p{16cm}|}{%
\begin{enumerate}\item Fundamentos da interpretabilidade e explicabilidade em IA.
\item Visualização de dados e modelos: técnicas e ferramentas.
\item Métodos locais e globais de interpretabilidade: LIME, SHAP, Partial Dependence Plots.
\item Visualização de redes neurais: camadas, pesos, gradientes e ativação.
\item Técnicas para explicar modelos complexos: árvores de decisão, ensembles e deep learning.
\item Interpretação de modelos para tomada de decisão responsável.
\item Viés e fairness em modelos de IA: identificação e mitigação.
\item Ferramentas e bibliotecas para interpretabilidade (InterpretML, Captum, ELI5).
\item Estudos de caso em setores como saúde, finanças e segurança.
\item Tendências e desafios futuros em interpretabilidade e visualização de IA.\end{enumerate}}\\
\multicolumn{4}{|p{16cm}|}{}\\
\multicolumn{4}{|p{16cm}|}{}\\
\multicolumn{4}{|p{16cm}|}{\vspace{-1cm}}\\
\multicolumn{4}{|p{16cm}|}{}\\
\hline
\multicolumn{4}{|p{16cm}|}{Bibliografia Básica:}\\
\multicolumn{4}{|p{16cm}|}{%
\begin{itemize}\item MOLNAR, Christoph. Interpretable Machine Learning. 1. ed. Lulu.com, 2019. Disponível em: https://christophm.github.io/interpretable-ml-book/. Acesso em: jun/2025.
\item CARUANA, Rich et al. Intelligible Models for Healthcare: Predicting Pneumonia Risk and Hospital 30-day Readmission. KDD, 2015.
\item LIPTON, Zachary C. The Mythos of Model Interpretability. Communications of the ACM, 2018.\end{itemize}}\\
\multicolumn{4}{|p{16cm}|}{}\\
\hline
\multicolumn{4}{|p{16cm}|}{Bibliografia Complementar:}\\
\multicolumn{4}{|p{16cm}|}{%
\begin{itemize}\item RIBEIRO, Marco Tulio et al. Why Should I Trust You? Explaining the Predictions of Any Classifier. KDD, 2016.
\item GUIDOTTI, Riccardo et al. A Survey of Methods for Explaining Black Box Models. ACM Computing Surveys, 2018.
\item DOSHI-VELEZ, Finale; KIM, Been. Towards A Rigorous Science of Interpretable Machine Learning. arXiv, 2017.
\item SAMEK, Wojciech et al. Explainable AI: Understanding, Visualizing and Interpreting Deep Learning Models. ITU Journal, 2019.
\item MENG, Xiao; CHEN, Ji. Visual Analytics for Explainable AI. Springer, 2021.\end{itemize}}\\
\hline
\end{longtable}
\end{center}

\clearpage
\begin{center}
\begin{tabular}{lccr}
 \multirow{3}{*}{\includegraphics[height=2.7cm]{brasao.png}} &
 \multicolumn{2}{c}{\bfseries UNIVERSIDADE FEDERAL DE OURO PRETO} &
 \ \ \ \ \multirow{3}{*}{\includegraphics[height=2.7cm]{ufop.png}} \\
 & \multicolumn{2}{c}{\bfseries PRÓ-REITORIA DE GRADUAÇÃO} & \\
 & \multicolumn{2}{c}{} & \\
 & \multicolumn{2}{c}{\Large\bfseries PROGRAMA DE DISCIPLINA} & \\
\end{tabular}
\end{center}

\begin{center}
\begin{longtable}{|p{4cm}|p{4cm}|p{4cm}|p{4cm}|}
\hline
\multicolumn{3}{|p{12cm}|}{Nome do Componente Curricular em Português:} &
\multicolumn{1}{p{4cm}|}{Código:} \\ 
\multicolumn{3}{|p{12cm}|}{\textbf{Projeto 1: Resolução de Problemas do Mundo Real}} &
\textbf{BIA101}\\ 
\multicolumn{3}{|p{12cm}|}{Nome do Componente Curricular em Inglês:} & \\ 
\multicolumn{3}{|p{12cm}|}{\textbf{Project 1: Real-World Problem Solving}} & \\ 
\hline
\multicolumn{3}{|p{12cm}|}{Nome e Sigla do Departamento} & Unidade Acadêmica: \\ 
\multicolumn{3}{|p{12cm}|}{Departamento de Computação (DECOM)} & {ICEB} \\ 
\hline
\multicolumn{4}{|p{16cm}|}{Modalidade de Oferta:
[X] presencial \hspace{1cm}
[ ] à distância}\\
\hline
\multicolumn{2}{|p{8cm}|}{Carga horária semestral} &
\multicolumn{2}{p{8cm}|}{Carga horária semanal}\\
\hline
\multicolumn{1}{|p{4cm}|}{Total} &
\multicolumn{1}{p{4cm}|}{Extensionista} &
\multicolumn{1}{p{4cm}|}{Teórica} &
\multicolumn{1}{p{4cm}|}{Prática} \\ 
\multicolumn{1}{|p{4cm}|}{60\,horas} &
\multicolumn{1}{p{4cm}|}{4\;horas} &
\multicolumn{1}{p{4cm}|}{0\;horas/aula} &
\multicolumn{1}{p{4cm}|}{0\;horas/aula} \\ 
\hline
\multicolumn{4}{|p{16cm}|}{Ementa:}\\
\multicolumn{4}{|p{16cm}|}{}\\
\multicolumn{4}{|p{16cm}|}{Desenvolvimento de projetos extensionistas voltados à identificação e resolução de problemas reais com aplicação de técnicas de inteligência artificial e computação. Enfoque em trabalho interdisciplinar, colaboração com a comunidade e inovação social.}\\
\multicolumn{4}{|p{16cm}|}{}\\
\hline
\multicolumn{4}{|p{16cm}|}{Conteúdo programático:}\\
\multicolumn{4}{|p{16cm}|}{%
\begin{enumerate}\item Identificação e definição de problemas reais em diferentes contextos sociais e econômicos.
\item Levantamento e análise das necessidades dos stakeholders.
\item Metodologias de pesquisa aplicada e extensão universitária.
\item Planejamento e gestão de projetos colaborativos.
\item Aplicação de técnicas de inteligência artificial e computação para desenvolvimento de soluções.
\item Desenvolvimento, implementação e prototipagem de projetos.
\item Avaliação de impacto social e técnico dos projetos.
\item Documentação e apresentação dos resultados.
\item Trabalho em equipe e comunicação interdisciplinar.
\item Ética, responsabilidade social e sustentabilidade em projetos tecnológicos.\end{enumerate}}\\
\multicolumn{4}{|p{16cm}|}{}\\
\multicolumn{4}{|p{16cm}|}{Perfil da Comunidade: A disciplina BIA101 tem como público-alvo comunidades locais que enfrentam desafios tecnológicos, promovendo uma troca de saberes entre a universidade e a sociedade. A atuação extensionista dos discentes será voltada para grupos que possam se beneficiar do uso da informática, incluindo escolas, organizações não-governamentais (ONGs), associações comunitárias e negócios locais. O foco da disciplina é permitir que os alunos utilizem seus conhecimentos para desenvolver soluções computacionais acessíveis, realizar capacitações e implementar ações concretas para ampliar o impacto social da computação. Essa interação direta possibilita que os discentes compreendam as demandas e limitações contextuais da sociedade, tornando-se agentes ativos na construção de uma tecnologia mais inclusiva e alinhada às necessidades reais da população.}\\
\multicolumn{4}{|p{16cm}|}{Objetivos Extensionistas: (i) Promover a inclusão digital e a democratização do acesso à tecnologia, desenvolvendo ações que auxiliem comunidades a superar barreiras tecnológicas e sociais. (ii) Capacitar os discentes para atuar como mediadores entre a universidade e a sociedade, aplicando conhecimentos computacionais para solucionar desafios reais enfrentados por escolas, ONGs, pequenos empreendedores e outras instituições comunitárias. (iii) Estimular a interação dialógica, garantindo que a construção das soluções tecnológicas ocorra de forma colaborativa, respeitando as demandas e o contexto da comunidade atendida. (iv) Incentivar a produção e a disseminação de conhecimento científico e tecnológico, promovendo a criação de materiais educativos, oficinas e treinamentos sobre informática e tecnologia acessível. (v) Avaliar o impacto social das ações desenvolvidas, garantindo que as soluções aplicadas gerem benefícios concretos e sustentáveis para a comunidade, ao mesmo tempo em que proporcionam uma experiência de aprendizagem significativa para os estudantes.}\\
\multicolumn{4}{|p{16cm}|}{}\\
\hline
\multicolumn{4}{|p{16cm}|}{Bibliografia Básica:}\\
\multicolumn{4}{|p{16cm}|}{%
\begin{itemize}\item SILVA, Maria Aparecida; LIMA, José Carlos. Metodologias de Pesquisa e Extensão. 1. ed. EDUFBA, 2019.
\item MORAIS, Regis de (org.). Filosofia da ciência e da tecnologia: introdução metodológica e crítica. Campinas: Papirus, 2013.
\item KELNER, John; JACOBSON, Steve. Project-Based Learning and Extension: A Guide to Community Engagement. Routledge, 2020.\end{itemize}}\\
\multicolumn{4}{|p{16cm}|}{}\\
\hline
\multicolumn{4}{|p{16cm}|}{Bibliografia Complementar:}\\
\multicolumn{4}{|p{16cm}|}{%
\begin{itemize}\item HECK, Angela; FERREIRA, Paulo. Inovação Social e Extensão Universitária. 1. ed. Editora UFSC, 2018.
\item BROWN, Tim. Change by Design: How Design Thinking Creates New Alternatives for Business and Society. HarperBusiness, 2009.
\item GIBSON, David. Managing Successful Projects with PRINCE2. 6. ed. The Stationery Office, 2017.
\item RUSSELL, Stuart; NORVIG, Peter. Inteligência Artificial. 3. ed. Pearson, 2013. (Capítulo sobre aplicações práticas).
\item MUNIZ, Antonio et al. Inteligência artificial: entenda como a IA pode impactar no mercado de trabalho e na sociedade. Brasport, 2024.\end{itemize}}\\
\hline
\end{longtable}
\end{center}

\clearpage
\begin{center}
\begin{tabular}{lccr}
 \multirow{3}{*}{\includegraphics[height=2.7cm]{brasao.png}} &
 \multicolumn{2}{c}{\bfseries UNIVERSIDADE FEDERAL DE OURO PRETO} &
 \ \ \ \ \multirow{3}{*}{\includegraphics[height=2.7cm]{ufop.png}} \\
 & \multicolumn{2}{c}{\bfseries PRÓ-REITORIA DE GRADUAÇÃO} & \\
 & \multicolumn{2}{c}{} & \\
 & \multicolumn{2}{c}{\Large\bfseries PROGRAMA DE DISCIPLINA} & \\
\end{tabular}
\end{center}

\begin{center}
\begin{longtable}{|p{4cm}|p{4cm}|p{4cm}|p{4cm}|}
\hline
\multicolumn{3}{|p{12cm}|}{Nome do Componente Curricular em Português:} &
\multicolumn{1}{p{4cm}|}{Código:} \\ 
\multicolumn{3}{|p{12cm}|}{\textbf{Projeto 2: Resolução de Problemas do Mundo Real}} &
\textbf{BIA102}\\ 
\multicolumn{3}{|p{12cm}|}{Nome do Componente Curricular em Inglês:} & \\ 
\multicolumn{3}{|p{12cm}|}{\textbf{Project 2: Real-World Problem Solving}} & \\ 
\hline
\multicolumn{3}{|p{12cm}|}{Nome e Sigla do Departamento} & Unidade Acadêmica: \\ 
\multicolumn{3}{|p{12cm}|}{Departamento de Computação (DECOM)} & {ICEB} \\ 
\hline
\multicolumn{4}{|p{16cm}|}{Modalidade de Oferta:
[X] presencial \hspace{1cm}
[ ] à distância}\\
\hline
\multicolumn{2}{|p{8cm}|}{Carga horária semestral} &
\multicolumn{2}{p{8cm}|}{Carga horária semanal}\\
\hline
\multicolumn{1}{|p{4cm}|}{Total} &
\multicolumn{1}{p{4cm}|}{Extensionista} &
\multicolumn{1}{p{4cm}|}{Teórica} &
\multicolumn{1}{p{4cm}|}{Prática} \\ 
\multicolumn{1}{|p{4cm}|}{60\,horas} &
\multicolumn{1}{p{4cm}|}{4\;horas} &
\multicolumn{1}{p{4cm}|}{0\;horas/aula} &
\multicolumn{1}{p{4cm}|}{0\;horas/aula} \\ 
\hline
\multicolumn{4}{|p{16cm}|}{Ementa:}\\
\multicolumn{4}{|p{16cm}|}{}\\
\multicolumn{4}{|p{16cm}|}{Continuação do desenvolvimento de projetos extensionistas focados na solução de problemas reais, com aprofundamento técnico e interdisciplinar. Ênfase em prototipagem avançada, validação e avaliação de impacto social e tecnológico.}\\
\multicolumn{4}{|p{16cm}|}{}\\
\hline
\multicolumn{4}{|p{16cm}|}{Conteúdo programático:}\\
\multicolumn{4}{|p{16cm}|}{%
\begin{enumerate}\item Revisão e aprimoramento dos projetos iniciados em Projeto 1.
\item Técnicas avançadas para desenvolvimento e otimização de soluções.
\item Validação e testes com usuários e stakeholders.
\item Metodologias para avaliação de impacto social e técnico.
\item Documentação técnica e relatórios de progresso.
\item Comunicação científica e apresentação de resultados.
\item Gestão de projetos e planejamento de etapas finais.
\item Ética e responsabilidade social no desenvolvimento tecnológico.
\item Trabalho colaborativo interdisciplinar e com a comunidade.
\item Preparação para continuidade e escalabilidade dos projetos.\end{enumerate}}\\
\multicolumn{4}{|p{16cm}|}{}\\
\multicolumn{4}{|p{16cm}|}{}\\
\multicolumn{4}{|p{16cm}|}{\vspace{-1cm}}\\
\multicolumn{4}{|p{16cm}|}{}\\
\hline
\multicolumn{4}{|p{16cm}|}{Bibliografia Básica:}\\
\multicolumn{4}{|p{16cm}|}{%
\begin{itemize}\item SILVA, Maria Aparecida; LIMA, José Carlos. Metodologias de Pesquisa e Extensão. 1. ed. EDUFBA, 2019.
\item MORAIS, Regis de (org.). Filosofia da ciência e da tecnologia: introdução metodológica e crítica. Campinas: Papirus, 2013.
\item KELNER, John; JACOBSON, Steve. Project-Based Learning and Extension: A Guide to Community Engagement. Routledge, 2020.\end{itemize}}\\
\multicolumn{4}{|p{16cm}|}{}\\
\hline
\multicolumn{4}{|p{16cm}|}{Bibliografia Complementar:}\\
\multicolumn{4}{|p{16cm}|}{%
\begin{itemize}\item HECK, Angela; FERREIRA, Paulo. Inovação Social e Extensão Universitária. 1. ed. Editora UFSC, 2018.
\item BROWN, Tim. Change by Design: How Design Thinking Creates New Alternatives for Business and Society. HarperBusiness, 2009.
\item GIBSON, David. Managing Successful Projects with PRINCE2. 6. ed. The Stationery Office, 2017.
\item RUSSELL, Stuart; NORVIG, Peter. Inteligência Artificial. 3. ed. Pearson, 2013. (Capítulo sobre aplicações práticas).
\item MUNIZ, Antonio et al. Inteligência artificial: entenda como a IA pode impactar no mercado de trabalho e na sociedade. Brasport, 2024.\end{itemize}}\\
\hline
\end{longtable}
\end{center}

\clearpage
\begin{center}
\begin{tabular}{lccr}
 \multirow{3}{*}{\includegraphics[height=2.7cm]{brasao.png}} &
 \multicolumn{2}{c}{\bfseries UNIVERSIDADE FEDERAL DE OURO PRETO} &
 \ \ \ \ \multirow{3}{*}{\includegraphics[height=2.7cm]{ufop.png}} \\
 & \multicolumn{2}{c}{\bfseries PRÓ-REITORIA DE GRADUAÇÃO} & \\
 & \multicolumn{2}{c}{} & \\
 & \multicolumn{2}{c}{\Large\bfseries PROGRAMA DE DISCIPLINA} & \\
\end{tabular}
\end{center}

\begin{center}
\begin{longtable}{|p{4cm}|p{4cm}|p{4cm}|p{4cm}|}
\hline
\multicolumn{3}{|p{12cm}|}{Nome do Componente Curricular em Português:} &
\multicolumn{1}{p{4cm}|}{Código:} \\ 
\multicolumn{3}{|p{12cm}|}{\textbf{Projeto 3: Resolução de Problemas do Mundo Real}} &
\textbf{BIA103}\\ 
\multicolumn{3}{|p{12cm}|}{Nome do Componente Curricular em Inglês:} & \\ 
\multicolumn{3}{|p{12cm}|}{\textbf{Project 3: Real-World Problem Solving}} & \\ 
\hline
\multicolumn{3}{|p{12cm}|}{Nome e Sigla do Departamento} & Unidade Acadêmica: \\ 
\multicolumn{3}{|p{12cm}|}{Departamento de Computação (DECOM)} & {ICEB} \\ 
\hline
\multicolumn{4}{|p{16cm}|}{Modalidade de Oferta:
[X] presencial \hspace{1cm}
[ ] à distância}\\
\hline
\multicolumn{2}{|p{8cm}|}{Carga horária semestral} &
\multicolumn{2}{p{8cm}|}{Carga horária semanal}\\
\hline
\multicolumn{1}{|p{4cm}|}{Total} &
\multicolumn{1}{p{4cm}|}{Extensionista} &
\multicolumn{1}{p{4cm}|}{Teórica} &
\multicolumn{1}{p{4cm}|}{Prática} \\ 
\multicolumn{1}{|p{4cm}|}{60\,horas} &
\multicolumn{1}{p{4cm}|}{4\;horas} &
\multicolumn{1}{p{4cm}|}{0\;horas/aula} &
\multicolumn{1}{p{4cm}|}{0\;horas/aula} \\ 
\hline
\multicolumn{4}{|p{16cm}|}{Ementa:}\\
\multicolumn{4}{|p{16cm}|}{}\\
\multicolumn{4}{|p{16cm}|}{Conclusão dos projetos extensionistas iniciados nas disciplinas anteriores, com foco na entrega final, avaliação de resultados, documentação completa e disseminação das soluções desenvolvidas para problemas reais.}\\
\multicolumn{4}{|p{16cm}|}{}\\
\hline
\multicolumn{4}{|p{16cm}|}{Conteúdo programático:}\\
\multicolumn{4}{|p{16cm}|}{%
\begin{enumerate}\item Finalização e refinamento das soluções desenvolvidas.
\item Validação definitiva com usuários e stakeholders.
\item Avaliação do impacto social, econômico e tecnológico.
\item Documentação técnica, científica e de extensão.
\item Preparação e realização de apresentações e defesas dos projetos.
\item Divulgação e publicação dos resultados em eventos e mídias acadêmicas e comunitárias.
\item Planejamento para continuidade, manutenção e escalabilidade das soluções.
\item Ética e responsabilidade na aplicação dos projetos.
\item Trabalho colaborativo interdisciplinar e comunitário.
\item Reflexão crítica sobre o processo de extensão e inovação social.\end{enumerate}}\\
\multicolumn{4}{|p{16cm}|}{}\\
\multicolumn{4}{|p{16cm}|}{}\\
\multicolumn{4}{|p{16cm}|}{\vspace{-1cm}}\\
\multicolumn{4}{|p{16cm}|}{}\\
\hline
\multicolumn{4}{|p{16cm}|}{Bibliografia Básica:}\\
\multicolumn{4}{|p{16cm}|}{%
\begin{itemize}\item SILVA, Maria Aparecida; LIMA, José Carlos. Metodologias de Pesquisa e Extensão. 1. ed. EDUFBA, 2019.
\item MORAIS, Regis de (org.). Filosofia da ciência e da tecnologia: introdução metodológica e crítica. Campinas: Papirus, 2013.
\item KELNER, John; JACOBSON, Steve. Project-Based Learning and Extension: A Guide to Community Engagement. Routledge, 2020.\end{itemize}}\\
\multicolumn{4}{|p{16cm}|}{}\\
\hline
\multicolumn{4}{|p{16cm}|}{Bibliografia Complementar:}\\
\multicolumn{4}{|p{16cm}|}{%
\begin{itemize}\item HECK, Angela; FERREIRA, Paulo. Inovação Social e Extensão Universitária. 1. ed. Editora UFSC, 2018.
\item BROWN, Tim. Change by Design: How Design Thinking Creates New Alternatives for Business and Society. HarperBusiness, 2009.
\item GIBSON, David. Managing Successful Projects with PRINCE2. 6. ed. The Stationery Office, 2017.
\item RUSSELL, Stuart; NORVIG, Peter. Inteligência Artificial. 3. ed. Pearson, 2013. (Capítulo sobre aplicações práticas).
\item MUNIZ, Antonio et al. Inteligência artificial: entenda como a IA pode impactar no mercado de trabalho e na sociedade. Brasport, 2024.\end{itemize}}\\
\hline
\end{longtable}
\end{center}

\clearpage
\begin{center}
\begin{tabular}{lccr}
 \multirow{3}{*}{\includegraphics[height=2.7cm]{brasao.png}} &
 \multicolumn{2}{c}{\bfseries UNIVERSIDADE FEDERAL DE OURO PRETO} &
 \ \ \ \ \multirow{3}{*}{\includegraphics[height=2.7cm]{ufop.png}} \\
 & \multicolumn{2}{c}{\bfseries PRÓ-REITORIA DE GRADUAÇÃO} & \\
 & \multicolumn{2}{c}{} & \\
 & \multicolumn{2}{c}{\Large\bfseries PROGRAMA DE DISCIPLINA} & \\
\end{tabular}
\end{center}

\begin{center}
\begin{longtable}{|p{4cm}|p{4cm}|p{4cm}|p{4cm}|}
\hline
\multicolumn{3}{|p{12cm}|}{Nome do Componente Curricular em Português:} &
\multicolumn{1}{p{4cm}|}{Código:} \\ 
\multicolumn{3}{|p{12cm}|}{\textbf{Projeto 4: Resolução de Problemas do Mundo Real}} &
\textbf{BIA104}\\ 
\multicolumn{3}{|p{12cm}|}{Nome do Componente Curricular em Inglês:} & \\ 
\multicolumn{3}{|p{12cm}|}{\textbf{Project 4: Real-World Problem Solving}} & \\ 
\hline
\multicolumn{3}{|p{12cm}|}{Nome e Sigla do Departamento} & Unidade Acadêmica: \\ 
\multicolumn{3}{|p{12cm}|}{Departamento de Computação (DECOM)} & {ICEB} \\ 
\hline
\multicolumn{4}{|p{16cm}|}{Modalidade de Oferta:
[X] presencial \hspace{1cm}
[ ] à distância}\\
\hline
\multicolumn{2}{|p{8cm}|}{Carga horária semestral} &
\multicolumn{2}{p{8cm}|}{Carga horária semanal}\\
\hline
\multicolumn{1}{|p{4cm}|}{Total} &
\multicolumn{1}{p{4cm}|}{Extensionista} &
\multicolumn{1}{p{4cm}|}{Teórica} &
\multicolumn{1}{p{4cm}|}{Prática} \\ 
\multicolumn{1}{|p{4cm}|}{60\,horas} &
\multicolumn{1}{p{4cm}|}{60\;horas} &
\multicolumn{1}{p{4cm}|}{0\;horas/aula} &
\multicolumn{1}{p{4cm}|}{4\;horas/aula} \\ 
\hline
\multicolumn{4}{|p{16cm}|}{Ementa:}\\
\multicolumn{4}{|p{16cm}|}{}\\
\multicolumn{4}{|p{16cm}|}{Desenvolvimento avançado de projetos extensionistas com foco na resolução de problemas reais, incorporando aprimoramentos técnicos, análise crítica dos resultados e estratégias para sustentabilidade e escalabilidade das soluções.}\\
\multicolumn{4}{|p{16cm}|}{}\\
\hline
\multicolumn{4}{|p{16cm}|}{Conteúdo programático:}\\
\multicolumn{4}{|p{16cm}|}{%
\begin{enumerate}\item Avaliação e análise crítica dos projetos anteriores.
\item Incorporação de melhorias técnicas e funcionais nas soluções.
\item Estudo de viabilidade para escalabilidade e sustentabilidade.
\item Integração de novas tecnologias e abordagens inovadoras.
\item Documentação atualizada e elaboração de relatórios finais.
\item Preparação para divulgação acadêmica e comunitária.
\item Planejamento estratégico para continuidade dos projetos.
\item Considerações éticas e responsabilidade social.
\item Trabalho colaborativo multidisciplinar e com a comunidade.
\item Reflexões sobre o impacto social e tecnológico dos projetos.\end{enumerate}}\\
\multicolumn{4}{|p{16cm}|}{}\\
\multicolumn{4}{|p{16cm}|}{}\\
\multicolumn{4}{|p{16cm}|}{\vspace{-1cm}}\\
\multicolumn{4}{|p{16cm}|}{}\\
\hline
\multicolumn{4}{|p{16cm}|}{Bibliografia Básica:}\\
\multicolumn{4}{|p{16cm}|}{%
\begin{itemize}\item SILVA, Maria Aparecida; LIMA, José Carlos. Metodologias de Pesquisa e Extensão. 1. ed. EDUFBA, 2019.
\item MORAIS, Regis de (org.). Filosofia da ciência e da tecnologia: introdução metodológica e crítica. Campinas: Papirus, 2013.
\item KELNER, John; JACOBSON, Steve. Project-Based Learning and Extension: A Guide to Community Engagement. Routledge, 2020.\end{itemize}}\\
\multicolumn{4}{|p{16cm}|}{}\\
\hline
\multicolumn{4}{|p{16cm}|}{Bibliografia Complementar:}\\
\multicolumn{4}{|p{16cm}|}{%
\begin{itemize}\item HECK, Angela; FERREIRA, Paulo. Inovação Social e Extensão Universitária. 1. ed. Editora UFSC, 2018.
\item BROWN, Tim. Change by Design: How Design Thinking Creates New Alternatives for Business and Society. HarperBusiness, 2009.
\item GIBSON, David. Managing Successful Projects with PRINCE2. 6. ed. The Stationery Office, 2017.
\item RUSSELL, Stuart; NORVIG, Peter. Inteligência Artificial. 3. ed. Pearson, 2013. (Capítulo sobre aplicações práticas).
\item MUNIZ, Antonio et al. Inteligência artificial: entenda como a IA pode impactar no mercado de trabalho e na sociedade. Brasport, 2024.\end{itemize}}\\
\hline
\end{longtable}
\end{center}

\clearpage
\end{document}